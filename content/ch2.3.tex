\chapter{初等变换与初等矩阵}

本节我们需要着重掌握初等变换与初等矩阵的相关内容,以及运用初等变换化矩阵为阶梯形矩阵,运用初等变换求逆矩阵。

\section{初等变换与初等矩阵}
初等变换是学习矩阵必须要掌握的一个基础内容,在后续内容中几乎都要用到初等变换,实际上初等变换就是我们解多元方程组的过程,即对某一方程同时乘以一个数,将两方程的顺序调换一下,将某两个方程变换消去一个未知量.
\subsection{初等变换}

矩阵的初等行变换:

(1)用非零常数$k$乘矩阵的某一行,

(2)交换矩阵某两行的位置,

(3)将某一行乘以$k$倍加到另外一行上去.

初等列变换同理,但我们一般都是用初等行变换来解决问题.

\subsection{初等矩阵}

1.初等矩阵及其逆矩阵:单位矩阵$E$经过一次初等行变换或一次初等列变换得到的矩阵,(注意是一次!)

第一种:交换$i$、$j$两行的位置,表示为$E_{ij}$;其逆矩阵就为该初等矩阵。

第二种:将第$i$行乘以非零常数k,表示为$E_i(k)$;其逆矩阵可通过单位矩阵第$i$行乘以$
	\frac{1}{k}$倍求得,表示为$E_{i}(\frac{1}{k})$。

第三种:将第$i$行乘以$k$倍加到第$j$行上,表示为$E_{ij}(k)$;其逆矩阵可通过单位矩阵第i行乘以$-k$倍加到第$j$行求得,表示为$E_{ij}(-k)$。

\begin{remark}
	一个可逆矩阵可以写成有限个初等矩阵的乘积,当然矩阵可逆的一个充分必要条件就是该矩阵可表示为有限个初等矩阵之积。
\end{remark}

2.“左行右列”:在明确初等矩阵是通过单位矩阵通过一次初等变换得到的之后,我们需要记住“左行右列”这一个口诀。左行右列指的是初等矩阵左乘一个矩阵,等于给该矩阵施加相应的初等行变换,初等矩阵右乘一个矩阵,等于给该矩阵施加相应的初等列变换。

\begin{example}
	设$A=\begin{pmatrix}
			a_{11} & a_{12} & a_{13} \\
			a_{21} & a_{22} & a_{23} \\
			a_{31} & a_{32} & a_{33}
		\end{pmatrix},B=\begin{pmatrix}
			a_{31} & a_{32} & a_{33}+a_{31} \\
			a_{21} & a_{22} & a_{23}+a_{21} \\
			a_{11} & a_{12} & a_{13}+a_{11}
		\end{pmatrix}.P_1=\begin{pmatrix}
			0 & 0 & 1 \\
			0 & 1 & 0 \\
			1 & 0 & 0
		\end{pmatrix},P_2=\begin{pmatrix}
			1 & 0 & 1 \\
			0 & 1 & 0 \\
			0 & 0 & 1
		\end{pmatrix}$,则$B=P_1AP_2$.
\end{example}

\begin{solution}
	我们通过观察可以发现$B$是由$A$先交换第一行和第三行,再将第一行加到第三行上去,分别对应先左乘$P_1$,再右乘$P_2$.
\end{solution}

\section{阶梯形矩阵}
阶梯形矩阵和行最简形可参考教材,不难理解,任何一个非零矩阵都可以通过初等变换化为阶梯形矩阵.
\subsection{化矩阵为阶梯形矩阵}

化矩阵为阶梯形矩阵实际上就是从第一行依此往下化简,然后在从第二行依次往下,依此类推,需要细心不出错,是我们后续解决问题的基本功.

\begin{example}
	化矩阵$\begin{pmatrix}
			2 & 1 & 2 & 4 \\
			1 & 2 & 1 & 2 \\
			3 & 0 & 2 & 2 \\
			2 & 1 & 4 & 5
		\end{pmatrix}$为阶梯形矩阵.
\end{example}

\begin{solution}
	一种思路,为了方便后续相减出现分数,我们将第一行与第二行交换,然后第二行减去第一行的两倍,第三行减去第一行的3倍,第四行减去第一行的两倍,第二行除以-3,第三行乘以-1,第三行减去第二行的6倍,第四行乘以-1,第四行减去第二行的3倍,第四行加上第三行,即可化为阶梯形矩阵.
\end{solution}

\section{初等变换求可逆矩阵}
求可逆矩阵通常有两种方法,一种是通过伴随矩阵求,一种就是通过初等行变换求,两者计算量可能都不小,看自己的喜好选择即可(推荐大于三阶以上使用初等行变换).

运用初等行变换:求$A$的逆矩阵,先在$A$的右侧写一个同阶的单位矩阵,通过初等变换将$A$化为单位矩阵,之前的单位矩阵变换后即是我们要求的矩阵.

即$[A|E]$通过初等行变换为$[E|A^{-1}]$.

例如
$$\left({\begin{array}{ccc:ccc}
			0  & 2  & -1 & 1 & 0 & 0 \\
			1  & 1  & 2  & 0 & 1 & 0 \\
			-1 & -1 & -1 & 0 & 0 & 1 \\
		\end{array}}\right)$$
可通过初等行变换化为
$$\left({\begin{array}{ccc:ccc}
			1 & 0 & 0 & -\frac{1}{2} & -\frac{3}{2} & -\frac{5}{2} \\
			0 & 1 & 0 & \frac{1}{2}  & \frac{1}{2}  & \frac{1}{2}  \\
			0 & 0 & 1 & 0            & 1            & 1
		\end{array}}\right)$$
则可求得$\begin{pmatrix}
		0  & 2  & -1 \\
		1  & 1  & 2  \\
		-1 & -1 & -1
	\end{pmatrix}$的逆矩阵为$\begin{pmatrix}
		-\frac{1}{2} & -\frac{3}{2} & -\frac{5}{2} \\
		\frac{1}{2}  & \frac{1}{2}  & \frac{1}{2}  \\
		0            & 1            & 1
	\end{pmatrix}$.

当然,我们也可以通过此方法求解矩阵方程$AX=B$,$X=A^{-1}B$.
即将$(A|B)$通过初等行变换化为$(E|A^{-1}B)$.

\section{等价矩阵}

\begin{enumerate}
	\item 若矩阵$A$可通过有限此初等行变换或有限次初等列变换得到矩阵$B$,则$A$和$B$等价,等价矩阵的秩相等。
	\item \begin{enumerate}
		\item $A$和$B$等价的充分必要条件:存在可逆矩阵$P$和$Q$,使得$PAQ=B$。
		\item $A$和$B$行等价的充分必要条件:存在可逆矩阵$P$,使得$PA=B$。
		\item $A$和$B$列等价的充分必要条件:存在可逆矩阵Q,使得$AQ=B$。
	\end{enumerate}
\end{enumerate}
