\chapter*{编写人员简介}
\addcontentsline{toc}{chapter}{编写\&排版人员简介}
\vspace{-3cm}

(按首拼字母排序)

\vspace{1em}

{\zihao{-2}\color{lbdeepblue}司秉钤}\quad 数试2101班学生,彭康学导团志愿者,在本份讲义中负责第二部分第2、3章编写。欢迎同学们来彭康学导团一起学习交流。同学们如有学习上的问题或者对本讲义有相关建议,请联系邮箱\url{2360804879@qq.com},希望和同学们一起学习进步。

\vspace{1em}

{\zihao{-2}\color{lbdeepblue}石学凯}\quad 智造钱2101班学生,彭康学导团成员,在本份讲义中负责第三部分第8、9章编写。希望和大家一起学习,无限进步。

\vspace{1em}

{\zihao{-2}\color{lbdeepblue}许祺}\quad 金禾2101班学生,彭康学导团热心志愿者,在本份讲义中负责第二部分第6章和第三部分第7章的编写。啥也不会,需要学习。对本讲义提出意见/勘误/合作请联系邮箱\url{2977038022@qq.com}。

\vspace{1em}
{\zihao{-2}\color{lbdeepblue}袁方}\quad 信息2105班学生,彭康学导团志愿者部成员,在本份讲义中负责第二部分第4、5章的编写。本人才疏学浅,所写不周之处还需要大家多多指正。最后希望大家认真学习,勤于思考,培养对数学学习的兴趣。

\vspace{1em}

{\zihao{-2}\color{lbdeepblue}袁思敏}\quad 大数据001班学生,在本份讲义中负责第一部分的编写,邮箱\url{1653867189@qq.com}。

\vspace{1em}

{\zihao{-2}\color{lbdeepblue}张恺}\quad 核工A002班学生,彭康学导团志愿者部部长,乐于分享,偏爱\LaTeX 排版,在本份讲义中负责全书的排版。自2021年加入彭康学导团大家庭以来,负责多份资料的编写和排版任务,包括高数等课程的真题集、《大学物理笔记(上、下)》、《流体力学·复习要点》等。

\vspace{2em}

在此,对以上牺牲个人宝贵时间来完成这份讲义的同学表示衷心感谢!