\chapter{矩阵的秩}

矩阵的秩,是线性代数的要义与精髓。但是课本的给出的定义直接用了“子式”的概念。虽然与矩阵以及行列式很相关,但是并没有体现线性代数的精髓。所以本节,我会先讲一些第四章“解方程组"相关的内容,然后再讲课本上的第二章第五节的内容。

\section{问题背景——解方程组}
\subsection{来源}
在中学中,我们接触过方程组。现在我们取一个有n个未知数,m个方程的线性方程组(方程组的常数项为0,如下)
$$ f(x)=\left\{
	\begin{aligned}
		a_{11}x_1+a_{12}x_2+...+a_{1n}x_n & = & 0 \\
		a_{21}x_1+a_{22}x_2+...+a_{2n}x_n & = & 0 \\
		\vdots                                    \\
		a_{m1}x_1+a_{m2}x_2+...+a_{mn}x_n & = & 0
	\end{aligned}
	\right.
$$

我们都知道,解方程组的时候,不是所有方程都可以作为一个单独的约束条件,有的方程是可以用其他几个方程推出来的,因此,我们需要研究方程真正的约束条件个数。
\subsection{研究步骤——取方程组的系数矩阵}
我们知道,上面的方程组中有意义的是每个未知数前面的系数,系数决定了方程组怎么解,有没有解,有多少解等诸多问题,我们将其提取出来,构成一个$m\times n$的矩阵:
$$
	\left (
	\begin{matrix}
			a_{11} & a_{12} & \cdots & a_{1n} \\
			a_{21} & a_{22} & \cdots & a_{2n} \\
			\vdots & \vdots & \ddots & \vdots \\
			a_{m1} & a_{m2} & \cdots & a_{mn} \\
		\end{matrix}
	\right  )
$$
因此我们得到:方程组的研究被转化为了矩阵的研究
\subsection{从方程组的有效约束条件——矩阵的秩}
我们都知道,方程组的有效约束条件$\leq$方程个数,而有效约束条件的个数,放在其系数矩阵中,就称为\textbf{矩阵的秩}。
\section{矩阵的秩的定义}
\subsection{矩阵的子式}
取一个$m\times n$的矩阵A,这是一个大矩阵。我们都学过分块矩阵,但是分块矩阵是对于一个矩阵的直接划分,切割形成了一些小矩阵出来。但是构造小矩阵不一定要用分割的方法,也可以提取某几行某几列,这就形成了子式的概念。
\begin{definition}
	对于矩阵$A$($m\times n$),任意抽取$k$行$k$列,按照原有的左右、上下的顺序组成一个$k$阶行列式,这个行列式称为\textbf{矩阵$A$的$k$阶子式}。
\end{definition}
\subsection{根本定义}
上面说了,矩阵的秩的出现,本身是为了解决方程组问题。但是作为线性代数本门学科,需要给出一个更普适性,以矩阵为基础的定义:
\begin{definition}
	矩阵$A$的不为零子式的最高阶数,称为矩阵的秩,记为$r(A)$。
\end{definition}
从此我们得到,我们可以从子式(行列式)的观点来学习矩阵的秩,也可以通过方程组进行理解。其中,方程组较为直观,但是可以推导的东西较少,主要作为一种辅助性的理解方式出现。
\subsection{定义分析}
我们知道,要求一个方程组的解(单一解/无数种解),无非要先用各种方法消除重复的条件,只保留有效约束的条件。而这一方法与前面矩阵化简化阶梯形的方法是一致的(这也正说明了矩阵与方程组的直接相通与一脉相承的关系)。

而由此我们得到:对于矩阵$A$,\textbf{矩阵的秩就是$A$的阶梯型矩阵的非零行个数$r$}。

这一点也可以由子式说明:矩阵的$r$个非零行与其$r$个首非零元,就已经构成了一个上三角行列式,此子式必定不为0。而一旦取得行数超过了$r$,就出现了零行,此子式一定为0。所以,矩阵的秩就是$r$\mn{后续讲义中,$r$直接表示$A$的秩。}。

\section{秩的结论推导——第一部分,简单性质推导}
\subsection{秩的有界性(名字是我自己起的)}
\begin{property}
	$0\leq r(A)\leq \min\{m,n\}$.
\end{property}
\begin{remark}
	\begin{itemize}
		\item 从方程组的角度上,有效行数肯定小于等于总行数;同时,有效行数肯定小于等于未知数个数。
		\item 从子式的角度上,显然,$k$不可能比总行数/总列数多。
	\end{itemize}
\end{remark}

\subsection{秩的转置不变性}
\begin{property}
	$r(A)=r(A^T)$.
\end{property}
\begin{remark}
	从子式的角度上:先取A的r阶子式$|B|$。在A转置后的$A^T$中,再按照原本的行看为列,原本的列看为行的方法,再把$|B|$的元素重新提取出来形成$|B^T|$。由于行列式的转置不变性,所以秩也有转置不变性。
\end{remark}
\subsection{当矩阵变为$n$阶方阵}
对$n$阶方阵,秩的最大值即为$n$。故:
$$r(A)=n\Leftrightarrow\det(A)\neq 0$$
此时,称$A$为满秩方阵(可逆方阵、非奇异方阵);否则,$A$为降秩方阵(不可逆方阵、奇异方阵)。
\section{秩的结论推导——第二部分,当矩阵遇见初等变换}
\subsection{初等变换等价性}
\begin{property}
	若矩阵$A$经过若干次初等变换变成了矩阵$B$,则$r(A)=r(B)$
\end{property}
\begin{remark}
	方程组角度:方程组进行的就是初等行变换,故初等行变换后自然有:秩不改变。又由秩的行列等价性,所以列变换后秩也不变。故:初等变换不改变矩阵的秩。
\end{remark}
\subsection{满秩方阵乘矩阵,秩不变}
\begin{property}
	设矩阵$A$、$P$、$Q$满足$A:m\times n$,$P:m\times m$可逆,$Q:n\times n$可逆,则有$r(PA)=r(A)$,$r(AQ)=r(A)$,$r(PAQ)=r(A)$。
\end{property}
\begin{remark}
	初等变换角度:满秩方阵可以分解为初等矩阵,由初等变化的秩不变性易证。
\end{remark}
\subsection{矩阵的秩标准形式}
任一非零矩阵$A$,可以化为左上角为单位矩阵,其余皆为0的矩阵$B$,矩阵$B$称为$A$的秩标准形,或者等价意义下的标准形,即必存在:$P:m\times m$可逆,$Q:n\times n$可逆,使得$PAQ= \left (
	\begin{matrix}
			I_r & 0 \\
			0   & 0 \\
		\end{matrix}
	\right  )$

此时,若$A_{m\times n}$秩为$m$(也称行满秩),则$A$的秩标准形为$$\left(	\begin{matrix}
			I_m & O_{m\times(n-m)} \\
		\end{matrix}
	\right)
$$

此时,若$A_{m\times n}$秩为$n$(也称列满秩),则$A$的秩标准形为:$$\left(\begin{matrix}
			I_m               & \\
			O_{(m-n)\times n} & \\
		\end{matrix}
	\right)
$$
\subsection{矩阵的满秩分解}
设$r(A_{m\times n})=r$,则存在列满秩矩阵$G_{m\times r}$和行满秩矩阵$H_{r\times n}$,使得$A=GH$其中,$r(G)=r(H)=r$。
\begin{proof}
	由矩阵的满秩形式,存在可逆矩阵$PQ$,使$$PAQ= \left (
		\begin{matrix}
				I_r & 0 \\
				0   & 0 \\
			\end{matrix}
		\right  )=\left(\begin{matrix}
				I_m & \\
				O   & \\
			\end{matrix}
		\right)
		\left(	\begin{matrix}
				I_m & O \\
			\end{matrix}
		\right)$$
	故左乘右乘过去之后,有:
	$$A= P^{-1}\left(\begin{matrix}
				I_m \\
				O   \\
			\end{matrix}
		\right)
		\left(	\begin{matrix}
				I_m & O \\
			\end{matrix}
		\right)Q^{-1}$$
	取$G= P^{-1}\left(\begin{matrix}
				I_m \\
				O   \\
			\end{matrix}
		\right),\mbox{取}H=	\left(	\begin{matrix}
				I_m & O \\
			\end{matrix}
		\right)Q^{-1}$
	证毕
\end{proof}
\section{关于矩阵的秩的二级结论}
注意,此部分应用到的知识点,新生应该还没有接触到,但是不用怕,学了相应的知识点后就会了。
\subsection{当你遇到$AB=0$}
对于矩阵$AB$,如果$AB=0$,则${\rm rank}(A)+{\rm rank}(B)\leq n$。
\begin{proof}
	取$W$为方程$Ax=0$的解空间,故$\dim(W)=n-{\rm rank}(A)$,而对于B,其任意列向量$\beta_i$都满足$\beta_i\in W(i=1,2,...,n)$,故${\rm rank}(B)\leq \dim(W)$,故$r(A)+r(B)\leq n$.
\end{proof}
\subsection{Sylvester不等式}
对于矩阵$A_{n\times n}$,$B_{n\times n}$,有${\rm rank}(AB)\geq {\rm rank}(A)+{\rm rank}(B)-n$。
\begin{proof}
	设${\rm rank}(A)=r$,易知$A$可以写为$P\left (
		\begin{matrix}
				I_r & O \\
				O   & O \\
			\end{matrix}
		\right  )Q$,故

	$$AB=P\left (
		\begin{matrix}
				I_r & O \\
				O   & O \\
			\end{matrix}
		\right  )QB$$
	考虑$QB=\left (
		\begin{matrix}
				H_1 \\
				H_2 \\
			\end{matrix}
		\right  )$,其中,$H_1:r\times n,H_2:r\times (n-r)\times n$。

	则出现
	$$AB=P\left (
		\begin{matrix}
				I_r & O \\
				O   & O \\
			\end{matrix}
		\right  )\left (
		\begin{matrix}
				H_1 \\
				H_2 \\
			\end{matrix}
		\right  )=P\left (
		\begin{matrix}
				H_1 \\
				0   \\
			\end{matrix}
		\right  )$$

	故${\rm rank}(AB)={\rm rank}\qty(\left (
		\begin{matrix}
				H_1 \\
				0   \\
			\end{matrix}
		\right  ))={\rm rank}(H_1),{\rm rank}(B)={\rm rank}(QB)={\rm rank}(\left (
		\begin{matrix}
				H_1 \\
				H_2 \\
			\end{matrix}
		\right  ))$,由${\rm rank}(H_1)+{\rm rank}(H_2){\rm rank}(H_1)={\rm rank}(AB)\geq {\rm rank}(B)+r-n$
	原式得证。
\end{proof}
\subsection{当遇见rank$(A+B)$}
$${\rm rank}(A+B)\leq {\rm rank}(A)+{\rm rank}(B)$$
\begin{proof}
	设$A$的列向量为$\alpha_i(i=1,\cdots,n)$,设$B$的列向量为$\beta_j(j=1,\cdots,n)$,取$\alpha_i,\cdots,\alpha_s$为$A$的极大线性无关组取$\beta_i,\cdots,\beta_r$为$B$的极大线性无关组,$A+B$的任意列向量可以由$<\alpha_i,\cdots,\alpha_s,\beta_i,\cdots,\beta_r>$线性表出,所以不等式成立。
\end{proof}
\subsection{伴随矩阵的秩}
$$ r(A^*)=\left\{
	\begin{aligned}
		n, & r(A)=n      \\
		1, & r(A)=n-1    \\
		0, & r(A)\le n-1 \\
	\end{aligned}
	\right.
$$
\begin{proof}
	由矩阵存在逆,故矩阵必然为$n\times n$方阵。

	I.若$r(A)=n=r(A^{-1})$,故由:$A^{*}=|A|A^{-1}$,得到$r(A^{*})=n$。

	II.若$r(A)=n-1$,故存在$n-1$阶子式不为0,故$r(A^*)\geq 1$,而$AA^*=|A|I=0$,$r(AA^*)+n=n\geq r(A)+r(A^*)$,故$r(A^*)\leq 1$,总体$r(A^*)=1$。

	III.若$r(A)\leq n-1,A^*=0$故$r(A^*)=0$。
\end{proof}

\section{模型、套路、题型}
\subsection{根据秩的数值,针对具体矩阵处理}
\begin{problem}
	设$n(n\geq 3)$阶矩阵$A=\left (
	\begin{matrix}
			1   & a   & a   & ... & a   \\
			a   & 1   & a   & ... & a   \\
			a   & a   & 1   & ... & a   \\
			... & ... & ... & ... & ... \\
			a   & a   & a   & ... & 1   \\
		\end{matrix}
	\right  )$,若$A$的秩为$n-1$,则$a = $?
\end{problem}
\begin{solution}
	不是满秩,因此$|A|=0$。
\begin{align*}
	|A|&=[(n-1)a+1]=\left |
	\begin{matrix}
		1   & a   & a   & ... & a   \\
		1   & 1   & a   & ... & a   \\
		1   & a   & 1   & ... & a   \\
		... & ... & ... & ... & ... \\
		1   & a   & a   & ... & 1   \\
	\end{matrix}
	\right  |=[(n-1)a+1]=\left |
	\begin{matrix}
		1   & a   & a   & ... & a   \\
		0   & 1-a & 0   & ... & 0   \\
		0   & 0   & 1-a & ... & 0   \\
		... & ... & ... & ... & ... \\
		0   & 0   & 0   & ... & 1-a \\
	\end{matrix}
	\right  |\\
	&=[(n-1)a+1](1-a)^{n-1}=0
\end{align*}

从而$a=\frac{1}{1-n}$或者$a=1$,若$a=1$,则$r(A)=1$,舍去,则$a=\frac{1}{1-n}$\mn{一般根据行列式为零会得到可能存在的数值。此时要依据秩的具体数值,逆,矩阵的特点来判定无效的矩阵。}。
\end{solution}

\subsection{初等变换求矩阵的秩}
\begin{problem}
	求矩阵$$\left (
	\begin{matrix}
			2  & -1 & 3       \\
			1  & -3 & 4       \\
			-1 & 2  & \lambda \\
		\end{matrix}
	\right  )$$的秩。
\end{problem}
\begin{solution}
	先化为阶梯型矩阵(原理:初等变换秩不变),然后阶梯型就直接根据定义看出来秩的数值了。
$$A=\left (
	\begin{matrix}
			2  & -1 & 3       \\
			1  & -3 & 4       \\
			-1 & 2  & \lambda \\
		\end{matrix}
	\right  )\xrightarrow{r_1\leftrightarrow r_2}\left (
	\begin{matrix}
			1  & -3 & 4       \\
			2  & -1 & 3       \\
			-1 & 2  & \lambda \\
		\end{matrix}
	\right  )\xrightarrow[r_3+r_1\times 1]{r_2+r_1\times(-2)} \left (
	\begin{matrix}
			1 & -3 & 4         \\
			0 & 5  & -5        \\
			0 & -1 & \lambda+4 \\
		\end{matrix}
	\right  )$$
$$\xrightarrow{r_2\times\frac{1}{5}}\left (
	\begin{matrix}
			1 & -3 & 4         \\
			0 & 1  & -1        \\
			0 & -1 & \lambda+4 \\
		\end{matrix}
	\right  )\xrightarrow{r_3+r_2}\left (
	\begin{matrix}
			1 & -3 & 4         \\
			0 & 1  & -1        \\
			0 & 0  & \lambda+3 \\
		\end{matrix}
	\right  )$$

当$\lambda=-3$时,有二阶子式不为0,所以$r(A)=2$;

当$\lambda\neq-3$时,有三阶子式不为0,所以$r(A)=3$。
\end{solution}

\subsection{根据具体的条件,处理某个性质的秩的证明题}
\begin{problem}
	证明对于任意一个秩为1的$n$阶方阵,$A$可以表示为$[a_1,a_2, \cdots,a_n]^{T}(b_1,b_2,...,b_n)$,且$A^2=kA$\mn{本题三个条件的关系常常出现在考试题里,一定要多留意。}。
\end{problem}
\begin{proof}
	由$r(A)=1$,必然其余行向量是某一行向量的倍数。
	\begin{enumerate}
		\item 不妨设$A=\left (
			\begin{matrix}
					a_1b_1 & a_1b_2 & ... & a_1b_n \\
					a_2b_1 & a_2b_2 & ... & a_2b_n \\
					...    & ...    & ... & ...    \\
					a_nb_1 & a_nb_2 & ... & a_nb_n \\
				\end{matrix}
			\right  )$,从而$A=[a_1,a_2,...,a_n]^T(b_1,b_2,...,b_n)$
		\item $A^2=[a_1,a_2,...,a_n]^T(b_1,b_2,...,b_n)[a_1,a_2,...,a_n]^T(b_1,b_2,...,b_n)=\sum_{i=1}^{n}a_ib_iA$,即$A^2=kA$
	\end{enumerate}
\end{proof}

\begin{problem}
	设A为$m\times n$矩阵,B为$n\times m$矩阵,证明:当$m\ge n$时,方阵$C=AB$不可逆。
\end{problem}
\begin{solution}
	因为$r(C)=r(AB)\leq\min{r(A),r(B)}\leq\min{m,n}$,实际$m\ge n$,故$\min{m,n}=n$,从而$m$阶方阵$C$来说,有$r(C)\leq\min{m,n}\le m$。
\end{solution}
\begin{remark}
	做题就是要不断考虑处理对象的性质,对于矩阵而言,包括:矩阵的大小$(m,n)$、秩、行列式、相应的方程的解与解空间等($Ax=0/Ax=b$)。这些性质或许与结论联系,或许方便构造,做题的过程要仔细思考它们。
\end{remark}

\begin{problem}
	设$A$是$n$阶方阵,且$A^2=E$,试证明:$r(A+E)+r(A-E)=n$。
\end{problem}
\begin{proof}
	对于题设结论,通过不同的处理得到不同形式,就会对应多个结论。
	$A^2-E=(A+E)(A-E)=O$,故$r(A+E)+r(A-E)\le n$
又有$r(A+E)+r(A-E)\ge r[(A+E)-(A-E)]=r(2E)=n$
\end{proof}
\begin{remark}
	本题的一部分结论会进行特殊化,比如如果$A$是二阶矩阵且$A\neq ±E$,那么$A+E$,$A-E$的秩就都是1。
\end{remark}

\begin{problem}
	设$n$阶方阵$A$,$B$满足$A^2=A$,$B^2=B$,且$E-A-B$可逆,证明:$r(A)=r(B)$。
\end{problem}
\begin{proof}
	题目中有一个好玩的条件:且$E-A-B$可逆。
	$A(E-A-B)=A-A^2-AB=-AB$,故$r(A)=r(AB)$,同理$r(B)=r(AB)$,故$r(A)=r(B)$。
\end{proof}
用关于秩的结论也可以证明,读者可以自行证明。