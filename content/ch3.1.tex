\chapter{向量及其代数运算}

高中的时候大家都学过向量,对于几何向量懂得不能再懂了。本章的向量更注重于向量的代数运算,即通过坐标法,将向量与空间的维度扩展,应用延长,创造一个基本的代数空间系统。

本节内容中,高中学过的就不再赘述了,只会讲一讲没有接触过的部分。考试一般不直接涉及本节的知识点,因为它们都是基础中的基础,也正因如此同学们也要重视起来这一部分。

\section{向量基本概念/向量线性运算}
本小节的知识点纯复习高中知识,全部略过,同学们自己复习一遍。

课本例3.1.1是高中定比分点的相关知识,同学们应该也很熟悉,课本有解析,不再赘述。

\section{向量共线,共面与线性表示}
两个向量$\vb*{a}$与$\vb*{b}$共线的充要条件是存在不全为零的常数$k_1$和$k_2$,使
$$k_1\vb*{a}+k_2\vb*{b}=\vb*{0}$$

三个向量$\vb*{a}$,$\vb*{b}$,$\vb*{c}$共面的充要条件使存在不全为0的常数$k_1,k_2,k_3$,使得
$$k_1\vb*{a}+k_2\vb*{b}+k_3\vb*{c}=\vb*{0}$$

设$\vb*{e_1},\vb*{e_2},\vb*{e_3}$是空间中不共面得3个向量,则空间中任一向量$\vb*{a}$都可以由$\vb*{e_1},\vb*{e_2},\vb*{e_3}$惟一地表示为:
$$\vb*{a}=x\vb*{e_1}+y\vb*{e_2}+z\vb*{e_3}$$
\section{空间坐标系与向量的坐标}
空间中有八个卦限,不多说。

通过坐标的的形式,向量可以表示为
$$\vb*{a}=(x,y,z)$$
\subsection{方向余弦}
(这个内容很容易理解,考试一般只考套公式)

在空间作图(或者高中知识)得到:$$\cos\alpha=\frac{x}{\sqrt{x^2+y^2+z^2}}$$
\subsection{用坐标表示向量共线与共面}
共线$\Leftrightarrow$坐标对应成比例。
共面$\Leftrightarrow$它们坐标构成的三阶行列式等于0,即:
$$
    \left |
    \begin{matrix}
        x_1 & x_2 & x_3 \\
        y_1 & y_2 & y_3 \\
        z_1 & z_2 & z_3 \\
    \end{matrix}
    \right  |=0
$$

正交与仿射部分的内容请同学们自行学习,学长记得老师好像略过了这一部分,但是同学们应该可以自己理解它们。

\section{模型、套路、题型}
作为向量概念的引入部分,本节做做课后题即可,没有需要特别强调的重点,也不是考试的重点。