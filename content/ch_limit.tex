\chapter{极限专题}

极限是高数的核心内容之一,本专题主要围绕极限,介绍求函数极限以及数列极限的方法,并总结一些相关题型。

极限的计算方法有很多种,最主要的几种就是利用等价无穷小替换、洛必达法则以及泰勒展开,会运用这些方法就能解决绝大多数关于极限的问题。其它的一些方法还有极限的定义、$Stolz$定理、夹逼准则、运用中值定理、运用已知极限(如重要极限)、导数的定义、定积分的定义、拟合法等等。本节内容主要介绍高数考试中会涉及到的方法。当然,一道极限题还需要用到一些常规的变换,如提取公因式、通分、换元(例如倒代换:$t = 1/x$)、取对数、因式分解、分子有理化等,并要求我们综合使用多种求极限的方法来求解。

本专题还总结了一些关于极限的题型,如已知一个极限求另一个极限、利用已知极限求参数、证明极限存在、关于递推数列的极限问题。

极限是考试的重点,考试中主要考查极限的求法,属于较好得分的内容。想要熟练掌握求极限的相关方法,需要多做一些习题、尝试一题多解(往往一道极限题可以运用多种不同的方法解决,此思想会在本节内容中反复出现),这样我们才能在处理极限问题上得心应手,游刃有余。

\section{求极限的三大方法}
\subsection{洛必达法则}
洛必达法则对于一些极限问题可谓是一大利器,当然,能否使用洛必达法则还需使用后才知道。使用洛必达法则需要明确分子分母同时趋于0或无穷,否则不能使用洛必达法则。对于一些较复杂的式子,则不推荐使用该法则。关于存在变限积分的极限题,往往需要使用洛必达法则,这是因为变限积分求导后会去掉积分符号。

\begin{example}
	$\lim_{x \to 1}\frac{\ln \cos(x-1)}{1-\sin \frac{\pi x}{2}}$
	\begin{solution}
		\begin{align*}
			\text{原式} & =\lim_{x \to 1}\frac{-\tan (x-1)}{-\frac{\pi}{2}\cos \frac{\pi x}{2}}     \\
			            & =\frac{2}{\pi}\lim_{x \to 1}\frac{x-1}{\cos \frac{\pi x}{2}}              \\
			            & =\frac{2}{\pi}\lim_{x \to 1}\frac{1}{-\frac{\pi}{2} \sin \frac{\pi x}{2}} \\
			            & =-\frac{4}{\pi^2}
		\end{align*}
	\end{solution}
\end{example}
\begin{remark}
	当$x \to 1$时,$\tan(x-1)\sim x-1$。
\end{remark}
\begin{example}
	$\lim_{x \to +\infty}\frac{\int_{0}^{x}(1+t^2)e^{t^2}dt}{xe^{x^2}+x^2}$
	\begin{solution}
		\begin{align*}
			\text{原式} & =\lim_{x \to +\infty}\frac{(1+x^2)e^{x^2}}{(2x^2+1)e^{x^2}+2x} \\
			            & =\lim_{x \to +\infty}\frac{1+x^2}{2x^2+1+\frac{2x}{e^{x^2}}}   \\
			            & =\frac{1}{2}
		\end{align*}
	\end{solution}
\end{example}
\begin{remark}
	变上限积分的求导:$\qty[\int_{a}^{f(x)}F(t){\rm d}t]' = f'(x)F(x)$,详细见积分章节内容。此题分子分母都趋向于$+\infty$,洛必达之后除以${\rm e}^{x^2}$化简后抓住最高阶项,即$x^2$。
\end{remark}

\subsection{等价无穷小的替换}
关于等价无穷小,我们需要明确两点,一是要记住并熟练运用常见的几种等价无穷小,二是不能随意运用等价无穷小,只能是乘除关系,不能是加减关系。实际上等价无穷小与泰勒公式有相通之处,在高中的导数题目中也经常出现类似的东西。

下面列出一些常见的等价无穷小及部分推导,帮助小伙伴们更好地理解和记忆。

当$x\to 0$时,
\begin{enumerate}
	\item $x \sim \sin x \sim \tan x \sim \arcsin x \sim \arctan x $;
	\item $e^x-1\sim x,\ln(1+x)\sim x,a^x-1 \sim x \ln a(a^x-1=e^{x \ln a} \sim x \ln a)$;
	\item $(1+x)^\alpha-1 \sim \alpha x$;
	\item $\cos x \sim 1-\frac{1}{2} x^2(1-\cos x \sim \frac{1}{2} x^2),x-\sin x \sim \frac{x^3}{6},x-\ln(1+x)\sim \frac{x^2}{2},\tan x-x\sim \frac{x^3}{3}$;
	\item $\tan x-\sin x \sim \frac{x^3}{2}(\tan x-\sin x=\tan x (1-\cos x)\sim x \cdot \frac{x^2}{2}=\frac{x^3}{2})$;
	\item $\arcsin x-x \sim \frac{x^3}{6},x-\arctan x \sim \frac{x^3}{3}$。
\end{enumerate}

\begin{remark}
	(1)对于代换4,可以结合下一部分的泰勒展开记忆;
	(2)对于代换6,由$x-\sin x \sim \frac{x^3}{6},x-\arcsin x=\arcsin (\sin x)\sim \frac{x^3}{6} \sim \frac{\sin^3 x}{6}$,则有$\arcsin x-x\sim \frac{x^3}{6}$,$x-\arctan x\sim \frac{x^3}{3}$的推导同理。
\end{remark}

另外需要注意到有时等价无穷小的替换不是很明显,需要我们注意式子的范围,进行加减凑项减项,例如$$f(x)\to 1\text{时},\ln f(x)=\ln [f(x)-1+1]\sim f(x)-1$$

\begin{example}
	$\lim_{x \to 0}\frac{(\tan x-\sin x)\ln(x+\sqrt{1+x^2})}{x^3\ln (1+\arctan x)}$
	\begin{solution}
		因为$\tan x-\sin x\sim \frac{x^3}{2}$,$\ln (x+\sqrt{1+x^2})=\ln (x+\sqrt{1+x^2}-1+1)\sim x+\sqrt{1+x^2}-1$,故$\ln (x+\sqrt{1+x^2})\sim x$,$\ln (1+\arctan x)\sim \arctan x \sim x$。

		$\text{原式}=\frac{\frac{x^3}{2}\cdot x}{x^3\cdot x}=\frac{1}{2}$
	\end{solution}
\end{example}

\subsection{泰勒展开}
运用泰勒展开来求极限可以说是十分万能的方法,只是是否简单明了。使用泰勒公式的关键,是确定展开的阶数,只需将最高阶数对应即可,通过做完一些题目我们不难理解其含义。

下面是需要记住的几个泰勒展开,

\begin{enumerate}
	\item $e^x=\sum_{k=0}^{n}\frac{x^k}{k!} +o(x^n)=1+x+\frac{x^2}{2}+o(x^2)$;
	\item $\ln(1+x)=\sum_{k=1}^{n}\frac{(-1)^{k-1}}{k}x^k +o(x^k)=x-\frac{x^2}{2}+o(x^2)$;
	\item $\sin x=\sum_{k=0}^{n}\frac{(-1)^k}{(2k+1)!}x^{2k+1} +o(x^{2k+1})=x-\frac{x^3}{6}+o(x^3)$;
	\item $\cos x=\sum_{k=0}^{n}\frac{(-1)^k}{(2k)!}x^{2k} +o(x^{2k})=1-\frac{x^2}{2}+o(x^2)$;
	\item $(1+x)^\alpha =1+\sum_{k=1}^{n}\frac{\alpha (\alpha-1) \cdots (\alpha-k+1)}{k!}x^k +o(x^k)=1+\alpha x+\frac{\alpha(\alpha-1)}{2}x^2+o(x^2)$;
	\item $\tan x=x+\frac{x^3}{3}+o(x^3)$;
	\item $(1+x)^{\frac{1}{x}}=e-\frac{e}{2}x+\frac{11}{24}ex^2+o(x^2)$。
\end{enumerate}

\begin{example}
	$\lim_{x \to 0}\frac{\frac{x^2}{2}+1-\sqrt{1+x^2}}{(\cos x -e^{x^2})\sin^2 x}$
	\begin{solution}
		因为$\sqrt{1+x^2}=1+\frac{1}{2}x^2+\frac{\frac{1}{2}(\frac{1}{2}-1)}{2!}(x^2)^2+o(x^4)$,$\cos x=1-\frac{1}{2}x^2+o(x^2)$,$e^{x^2}=1+x^2+o(x^2)$。

		故\begin{align*}
			\text{原式} & = \lim_{x\to 0}\frac{\frac{1}{2}x^2+1-(1+\frac{1}{2}x^2-\frac{1}{8}x^4+o(x^4))}{[1-\frac{1}{2}x^2-(1+x^2)+o(x^2)]x^2} \\
			            & =\lim_{x \to 0}\frac{\frac{1}{8}x^4+o(x^4)}{-\frac{3}{2}x^4+o(x^4)}=-\frac{1}{12}
		\end{align*}
		.
	\end{solution}
\end{example}

\section{求极限进阶}
本小节内容不刻意将求极限题目按照特定的方法划分(具体运用到的一些方法会随题目引出),而是追求对于一些类型的题目进行一题多解,体会多种不同的方法的使用。
\subsection{$n$项相加或相乘}
对于求$n$项连加形式的极限,主要是夹逼法则和定积分的定义两个方法,或者直接求和;对于$n$项连乘形式的极限,通常是取对数化成$n$项相加的形式,或者凑一个式子连消。另外,对于含有$n$个的相关题目也可以考虑差分法(在下面会有题目进行介绍)。

\begin{example}
	$\lim_{n \to \infty}(\frac{1}{n+1}+\frac{1}{n+2}+\cdots +\frac{1}{n+n})$
\end{example}
\begin{solution}
	\begin{enumerate}
		\item 法一(定积分定义)
		      \begin{align*}
			      \text{原式} & =\lim_{n \to \infty}\frac{1}{n}(\frac{1}{1+\frac{1}{n}}+\frac{1}{1+\frac{2}{n}}+\cdots+\frac{1}{1+\frac{n}{n}}) \\=\int_{0}^{1}\frac{1}{1+x}dx=\ln (1+x)|_{0}^{1}=\ln 2
		      \end{align*}

		      其实对于$\sum\frac{T}{S}$形式的极限题($S,T$为多项式),都可以用定积分的定义来做,对于$S$齐次,$T$齐次且$S$比$T$高一次的情形,可以直接转化为$\frac{1}{n}\sum f(\frac{k}{n})$的形式,不满足此条件的可以通过拟合法,间接求。详情可参考\url{https://b23.tv/bO4Aoje}。

		      另外,还有一类题,积分与求和同时出现,此类问题具有一定难度,本质上是积分与求和的统一,通常是利用积分区间的可加性或积分放缩法,将积分化成求和的形式或将求和化为积分的形式,在积分章节中我们需要注意一下此类问题。例如,已知$f(x)$在$[0,+\infty ]$上为非负的连续函数,且$f^{'}(x) \leq 0$,$a_n=\sum_{k=1}^{n}f(k)-\int_{1}^{n}f(x)dx$,证明$\{a_n\}$收敛.
		\item 法2(运用欧拉常数)
		      欧拉常数:$C=\lim_{n \to \infty}(1+\frac{1}{2}+\cdots+\frac{1}{n}-\ln n)$\\
		      原式$=\lim_{n \to \infty}[(1+\frac{1}{2}+\cdots+\frac{1}{2n})-(1+\frac{1}{2}+\cdots+\frac{1}{n}-\ln n)]$\\
		      $=\lim_{n \to \infty}[(1+\frac{1}{2}+\cdots+\frac{1}{2n}-\ln 2n)-(1+\frac{1}{2}+\cdots+\frac{1}{n}-\ln n-
				      \ln n)+\ln 2n -\ln n]$\\
		      $=C-C+\ln \frac{2n}{n}=\ln 2$.
		\item 法3(夹逼准则)
		      推荐记住这个不等式:$$\frac{1}{n+1}<\ln(1+\frac{1}{n})<\frac{1}{n}.$$

		      由上述不等式有:
		      $\begin{cases}
				      $$\frac{1}{n+1} <\ln(1+\frac{1}{n})<\frac{1}{n}.(1)$$        \\
				      $$\frac{1}{n+2}<\ln(1+\frac{1}{n+1})<\frac{1}{n}.(2)$$       \\
				      $$\cdots$$                                                   \\
				      $$\frac{1}{n+n}<\ln(1+\frac{1}{n+n-1})<\frac{1}{n+n-1}.(n)$$ \\
				      $$\frac{1}{n+n+1}<\ln(1+\frac{1}{n+n})<\frac{1}{n+n}.(n+1)$$ \\\end{cases}$\\
		      由前$n$个式子有:$\frac{1}{n+1}+\frac{1}{n+2}+\cdots +\frac{1}{n+n}<\ln
			      (\frac{n+1}{n}\cdot\frac{n+2}{n+1}\cdot\cdots\cdot\frac{n+n}{n+n-1})=\ln 2$,\\
		      由后$n$个式子有:$\frac{1}{n+1}+\frac{1}{n+2}+\cdots +\frac{1}{n+n}>\ln
			      (\frac{n+2}{n+1}\cdot\frac{n+3}{n+2}\cdot\cdots\cdot\frac{n+n+1}{n+n})=\ln \frac{2n+1}{n+1}\to \ln 2$.\\
		      由夹逼准则有:$\lim_{n \to \infty}(\frac{1}{n+1}+\frac{1}{n+2}+\cdots +\frac{1}{n+n})=\ln 2$
	\end{enumerate}
\end{solution}
\begin{example}
	$\lim_{x \to \infty}(\frac{\sin \frac{\pi}{n}}{n+\frac{1}{n}}+\frac{\sin \frac{2\pi}{n}}{n+\frac{2}{n}}+\cdots+\frac{\sin \frac{n\pi}{n}}{n+\frac{n}{n}})$
	\begin{solution}
		原式$\geq \frac{1}{n+1}(\sin \frac{\pi}{n}+\sin \frac{2\pi}{n}+\cdots+\sin \frac{n\pi}{n}),$\\
		$\frac{1}{n+1}(\sin \frac{\pi}{n}+\sin \frac{2\pi}{n}+\cdots+\sin \frac{n\pi}{n})=\frac{n}{n+1}\cdot \frac{1}{n}(\sin \frac{\pi}{n}+\sin \frac{2\pi}{n}+\cdots+\sin \frac{n\pi}{n})\to \frac{n}{n+1}\int_{0}^{1}\sin\pi xdx \to \frac{2}{\pi}$;\\
		原式$\leq \frac{1}{n}(\sin \frac{\pi}{n}+\sin \frac{2\pi}{n}+\cdots+\sin \frac{n\pi}{n})$,\\
		$\frac{1}{n}(\sin \frac{\pi}{n}+\sin \frac{2\pi}{n}+\cdots+\sin \frac{n\pi}{n})\to \int_{0}^{1}\sin\pi xdx = \frac{2}{\pi}$.\\
		由夹逼准则有,原式=$\frac{2}{\pi}$.
	\end{solution}
\end{example}

\begin{example}
	(1)$\lim_{x \to \infty}(1+a)(1+a^2)\cdots (1+a^{2^n}),|a|<1$ \\
	(2)$\lim_{x \to \infty}\cos\frac{\theta}{2}\cdot \cos\frac{\theta}{2^2}\cdots \cos\frac{\theta}{2^n}$($\theta \neq 0$)
	\begin{solution}
		(1)原式$=\lim_{n \to \infty}\frac{(1-a)(1+a)(1+a^2)\cdots (1+a^{2^n})}{1-a}$\\
		$=\lim_{n \to \infty}\frac{1-a^{2^{n+1}}}{1-a}$\\
		$=\frac{1}{1-a}$.\\
		(2)原式$=\lim_{n \to \infty}\frac{\sin \frac{\theta}{2^n}\cos\frac{\theta}{2}\cdot \cos\frac{\theta}{2^2}\cdots \cos\frac{\theta}{2^n}}{\sin \frac{\theta}{2^n}}$\\
		$=\lim_{n \to \infty}\frac{sin \theta}{2^n\sin \frac{\theta}{2^n}}$\\
		$=\frac{\sin \theta}{\theta}$.
	\end{solution}
\end{example}
\begin{example}
	$\lim_{x \to 0}\frac{1-\cos x \sqrt{\cos2x} \cdots \sqrt[n]{\cos nx}}{x^2}$
\end{example}
\begin{solution}
	\begin{enumerate}
		\item 法1(洛必达)
		      本题运用洛必达法则之前需要明确一点,$f_1(x)f_2(x)\cdots f_n(x)$的导数为$\sum_{k=1}^{n}[f_k(x)]^{'}(\cdots)$,即每一项的导数与其它项的乘积的和,由求导法则不难推导.\\
		      原式$=-\lim_{x \to 0}\frac{(\cos x)^{'}(\cdots)+(\sqrt{\cos 2x})^{'}(\cdots)+\cdots+(\sqrt[n]{\cos nx})(\cdots)}{2x}$\\
		      把分子单独拿出来,\\$(\cos x)^{'}(\cdots)+(\sqrt{\cos 2x})^{'}(\cdots)+\cdots+(\sqrt[n]{\cos nx})(\cdots)$\\
		      $=\cos x \sqrt{\cos2x} \cdots \sqrt[n]{\cos nx}\sum_{k=1}^{n}\frac{(\sqrt[k]{\cos kx})^{'}}{\sqrt[k]{\cos kx}}$\\
		      $=\sum_{k=1}^{n}(\sqrt[k]{\cos kx})^{'}$.\\
			      故原式$=-\lim_{x \to 0}\frac{\sum_{k=1}^{n}(\sqrt[k]{\cos kx})^{'}}{2x}$\\
		      $=\sum_{k=1}^{n}\lim_{x \to 0}\frac{\sin kx}{2x}$\\
		      $=\sum_{k=1}^{n}\frac{k}{2}=\frac{1}{2}\frac{n(n+1)}{2}=\frac{n(n+1)}{4}$
		\item 法2(等价无穷小)
		      原式$=\lim_{x \to 0}\frac{1-e^{\ln \cos x \sqrt{\cos2x} \cdots \sqrt[n]{\cos nx}}}{x^2}$\\
		      $=-\lim_{x \to 0}\frac{-\ln \cos x \sqrt{\cos2x} \cdots \sqrt[n]{\cos nx}}{x^2}$\\
		      $=-\lim_{x \to 0}\frac{\sum_{k=1}^{n}\ln \sqrt[k]{\cos kx}}{x^2}$,\\
		      由于原式$\ln \sqrt[k]{\cos kx}=\frac{1}{k}\ln\cos kx\sim \frac{1}{k}(\cos kx-1)\sim \frac{1}{k}[-\frac{1}{2}(kx)^2]=-\frac{k}{2}x^2$,\\
		      所以$=-\sum_{k=1}^{n}\lim_{x \to 0}\frac{-\frac{k}{2}x^2}{x^2}$\\
		      $=\sum_{k=1}^{n}\frac{k}{2}=\frac{n(n+1)}{4}$.

		      注:本题解法蕴含着对$n$项相乘取对数化为$n$项相加的思想.
		\item 法3(泰勒展开)
		      一般形式:$\sqrt[k]{\cos kx}=\sqrt[k]{\cos kx -1+1}-1+1=\frac{1}{k}(\cos kx-1)+1+o(\cos kx-1)=1-\frac{kx^2}{2}+o(x^2)$.\\
		      则有$\cos x \sqrt{\cos2x} \cdots \sqrt[n]{\cos nx}$\\
		      $=(1-\frac{x^2}{2}+o(x^2))(1-\frac{2x^2}{2}+o(x^2))\cdots(1-\frac{nx^2}{2}+o(x^2))$\\
		      $=1-\frac{1}{2}x^2-x^2-\frac{n}{2}x^2+o(x^2)$,代入得:\\
		      原式$=\lim_{x \to 0}\frac{1-(1-\frac{1}{2}x^2-x^2-\frac{n}{2}x^2+o(x^2))}{x^2}$\\
		      $=\frac{1}{2}+\frac{2}{2}+\cdots+\frac{n}{2}=\frac{n(n+1)}{2}=\frac{n(n+1)}{4}$.
		\item 法4(差分法)
		      在运用差分法之前,我们先对差分法做一个介绍,高数并未涉及差分方程的知识,因此对于差分我们有一个简单的了解即可,实际上在高中数列中我们就有所接触。差分法:$$I_n=\sum_{k=2}^{n}(I_k-I_{k-1})+I_1$$具体如何应用请看实际操作。\\
		      令$I_n=\frac{1-\cos x \sqrt{\cos2x} \cdots \sqrt[n]{\cos nx}}{x^2}$,\\
		      $I_k-I_{k-1}=-\frac{\cos x \sqrt{\cos2x} \cdots \sqrt[k-1]{\cos (k-1)x}(\sqrt[k]{\cos kx}-1)}{x^2}$\\
		      $=-\frac{\sqrt[k]{\cos kx}-1+1}{x^2}\sim \frac{1-\cos kx}{kx^2}\sim \frac{k}{2}$\\
		      由此可得$I_n=\sum_{k=2}^{n}(I_k-I_{k-1})+I_1$\\
		      $=\sum_{k=2}^{n}\frac{k}{2}+\lim_{x \to 0}\frac{1-\cos x}{x^2}$\\
		      $=\sum_{k=1}^{n}\frac{k}{2}$\\
		      $=\frac{n(n+1)}{4}$.
		\item 法5(凑项法)
		      原式$=\lim_{x \to 0}\frac{1-\cos x+\cos x-\cos x\sqrt{\cos 2x}+\cos x\sqrt{\cos 2x}\cdots-\cos x \sqrt{\cos2x} \cdots \sqrt[n]{\cos nx}}{x^2}$\\
		      $=\frac{1}{2}+\frac{2}{2}+\cdots+\frac{n}{2}$\\
		      $=\frac{n(n+1)}{4}$.

		      注:我们可以看到,凑项法与差分法的本质其实是一样的,因此,过程不详细写,小伙伴们可自行补充.
	\end{enumerate}
\end{solution}

\subsection{幂指形式}
幂指形式的极限求法,主要有两种,一是化成重要极限的形式,二是取对数化成$e^{\ln f(x)}$的形式。

\begin{example}
	$\lim_{x \to \infty}(\frac{2^{\frac{1}{x}}+53^{\frac{1}{x}}+49069^{\frac{1}{x}}}{3})^{3x}$
\end{example}\
\begin{solution}
	\begin{enumerate}
		\item 法一
		      原式$=\lim_{x \to 0}(\frac{2^{x}+53^{x}+49069^{x}}{3})^{\frac{3}{x}}$ (倒代换:用$x$替换$\frac{1}{x}$,$x \to 0$)  \\ $=e^{\lim_{x \to 0}3 \frac{\ln(2^x+53^x+49069^x)-\ln 3}{x}}$ (取对数)\\$=e^{\lim_{x \to 0}3 \frac{2^x\ln2+53^x\ln 53+49069^x\ln 49069}{2^x+53^x+49069^x}}$ (分子分母趋向于0,运用洛必达法则)\\
		      $=e^{\ln (2*53*49069)}=5201314$

		\item 法二
		      原式$=\lim_{x \to 0}(\frac{2^{x}+53^{x}+49069^{x}}{3})^{\frac{3}{x}}$ (倒代换:用$x$替换$\frac{1}{x}$,$x \to 0$)\\
		      $=\lim_{x \to 0}(1+\frac{2^x+53^x+49069^x-3}{3})^{\frac{3}{2^x+53^x+49069^x-3}\frac{2^x+53^x+49069^x-3}{x}}$ (凑成重要极限的形式)\\
		      $=e^{\lim_{x \to 0}\frac{2^x+53^x+49069^x-3}{x}}=e^{\lim_{x \to 0}2^x\ln2+53^x\ln 53+49069^x\ln 49069}$  (洛必达)\\
		      $=e^{\ln (2*53*49069)}=5201314$
	\end{enumerate}
\end{solution}
\begin{example}
	$\lim_{n \to \infty}(n\sin \frac{1}{n})^{n^2}$
	\begin{solution}
		倒代换,取对数,等价无穷小替换(洛必达法则也可).\\
		原式$=\lim_{x \to 0}(\frac{1}{x}\sin x)^{\frac{1}{x^2}}$\\
		$=e^{\lim_{x \to 0}\frac{\ln \frac{\sin x}{x}}{x^2}}$\\
		$=e^{\lim_{x \to 0}\frac{\frac{\sin x}{x}-1}{x^2}}$\\
		$=e^{\lim_{x \to 0}\frac{\sin x-x}{x^2}}$\\
		$=e^{\lim_{x \to 0}\frac{-\frac{1}{6}x^3}{x^3}}$\\$=e^{-\frac{1}{6}}$
	\end{solution}
\end{example}

\begin{example}
	$\lim_{x \to +\infty}[\sqrt{n}(\sqrt{n+1}-\sqrt{n})+\frac{1}{2}]^{\frac{\sqrt{n+1}+\sqrt{n}}{\sqrt{n+1}-\sqrt{n}}}$
	\begin{solution}
		取对数并通分化简,原式$=\lim_{n \to +\infty}e^{\frac{\sqrt{n+1}+\sqrt{n}}{\sqrt{n+1}-\sqrt{n}}\ln (\frac{\sqrt{n}-\sqrt{n+1}}{2(\sqrt{n+1}+\sqrt{n})}+1)}$,\\
		等价无穷小替换,得原式$=e^{-\frac{1}{2}}$。
	\end{solution}
\end{example}

\subsection{$\infty - \infty$类型}
$\infty - \infty$类型的极限,可以直接通分、填项凑项或用倒代换($t=\frac{1}{x}$),拉格朗日中值定理等方法。
\begin{example}
	$\lim_{x \to +\infty}(\sqrt{x+\sqrt{x+\sqrt{x}}}-\sqrt{x})$
\end{example}
\begin{solution}
	\begin{enumerate}
		\item 法一
		      原式$=\lim_{n \to +\infty}\frac{\sqrt{x+\sqrt{x}}}{\sqrt{x+\sqrt{x+\sqrt{x}}}+\sqrt{x}}\text{有理化}$\\
		      $=\lim_{x \to +\infty}\frac{\sqrt{1+\frac{1}{\sqrt{x}}}}{\sqrt{1+\sqrt{\frac{1}{x}+\frac{1}{x\sqrt{x}}}}+1}$\\
		      =$\frac{1}{2}$
		\item 法二
		      原式$=\lim_{n \to +\infty}\sqrt{x}(\sqrt {1+\sqrt{\frac{1}{x}+\frac{1}{x\sqrt{x}}}}-1)$\\
		      $=\lim_{n \to +\infty}\sqrt{x}\cdot \frac{1}{2}\sqrt{\frac{1}{x}+\frac{1}{x\sqrt{x}}}.$(等价无穷小替换)\\
		      $=\lim_{n \to +\infty}\frac{1}{2}\sqrt{1+\frac{1}{\sqrt{x}}}$\\
		      $=\frac{1}{2}$。
	\end{enumerate}
\end{solution}
\begin{example}
	$\lim_{x \to +\infty}(\sqrt{x^3+2x^2+1}-xe^{\frac{1}{x}})$
	\begin{solution}
		本题我们可以观察到既有三次根号,又有$e^{\frac{1}{x}}$,不易通过普通方法求解,我们可以看到
		$e^{\frac{1}{x}}$是趋于1的,因此我们可以考虑使用拟合法,因为$\sqrt{x^3+2x^2+1}-x$的极限是比较好求的,拟合法具体操作见下.

		原式$=\lim_{x \to +\infty}(\sqrt{x^3+2x^2+1}-x+x-xe^{\frac{1}{x}})$\\
		$=\lim_{x \to +\infty}(\sqrt{x^3+2x^2+1}-x)+\lim_{x \to +\infty}x(1-e^{\frac{1}{x}})$\\
		即将一个不易求的极限通过拟合成一个易求的极限加上另一个极限,前提是拟合出来的极限必须都存在.\\
		接着通过倒代换和等价无穷小的替换,求出两个极限.\\
		$=\lim_{x \to +\infty}(\sqrt{x^3+2x^2+1}-x)+\lim_{x \to +\infty}x(1-e^{\frac{1}{x}})$\\
		$=\lim_{t \to 0^+}\frac{\sqrt[3]{1+2t+t^3}-1}{t}+\lim_{x \to +\infty}x\cdot (-\frac{1}{x})$\\
		$=\lim_{t \to 0^+}\frac{\frac{1}{3}(2t+t^3)}{t}-1$\\
		$=-\frac{1}{3}$.
	\end{solution}
\end{example}

\subsection{运用连续性}
运用到连续性的题目,不容易通过其它方法做出来,需要对原式进行变换,目前见过的题型就是运用三角函数的周期性转换,再根据初等函数在有定义的地方皆连续得出结果,关键是前者。
\begin{example}
	$\lim_{x \to \infty}\sin ^2(\pi \sqrt{n^2+n})$
	\begin{solution}
		由于$$\sin^2(\pi \sqrt{n^2+n})=\sin^2(\pi \sqrt{n^2+n}-n\pi)=\sin^2\frac{n\pi}{\sqrt{n^2+n}+n}=sin^2\frac{\pi}{\sqrt{1+\frac{1}{n}}+1}.$$由于初等函数在有定义的地方都连续,得原式$=sin^2(\lim_{x \to \infty}\frac{\pi}{\sqrt{1+\frac{1}{n}}+1})=\sin^2\frac{\pi}{2}=1$。
	\end{solution}
\end{example}
\textbf{例:}

\textit{解:}

\begin{remark}
	函数$y=sin^x$的周期为$n\pi$,所以$\sin^2(\pi \sqrt{n^2+n})=\sin^2(\pi \sqrt{n^2+n}-n\pi)$。
\end{remark}

其它类似的两道题:

(1)$\lim_{x \to \infty}n\tan \pi \sqrt{n^2+1}$.

(2)$\lim_{x \to \infty}(1+\sin \pi \sqrt{1+4n^2})^n$.

这两道题处理方法与例题类似,关键点在于:

(1)$\tan \pi \sqrt{n^2+1} =\tan (\pi \sqrt{n^2+1}-n\pi)$

(2)$\sin\pi\sqrt{1+4n^2}=\sin(\sqrt{1+4n^2}-2n\pi)$

答案分别为$\frac{\pi}{2}$,$e^{\frac{\pi}{4}}$\mn{思考:为什么分别按$n \pi$,$2n\pi$进行周期变换?}。

\subsection{拉格朗日中值定理求极限}

拉格朗日中值定理在求极限中的应用(同名函数相减):$\exists \xi \in (f(x),g(x))$,\\	$F(f(x))-F(g(x))=F^{'}(\xi)(f(x)-g(x))$,往往$f(x)$与$g(x)$都同时趋向于一个数。

当然,对于非同名函数相减,我们可以根据具体情形转化为同名函数相减。

\begin{example}
	$\lim_{x \to +\infty}(\sin \sqrt{x+1}-\sin \sqrt{x})$
	\begin{solution}
		\begin{enumerate}
			\item 法1(和差化积)
			      原式$=\lim_{n \to +\infty}2\cos \frac{\sqrt{x+1}+\sqrt{x}}{2}\sin \frac{1}{2(\sqrt{x+1}+\sqrt{x})}=0$.
			\item 法2(中值定理)
			      原式$=\lim_{x \to +\infty}\cos \xi(\sqrt{x+1}-\sqrt{x})=\lim_{n \to +\infty}\cos \xi \frac{1}{\sqrt{x+1}+\sqrt{x}}=0$.
		\end{enumerate}
	\end{solution}
\end{example}

\begin{example}
	$\lim_{x \to 0}\frac{1-\ln[\ln(x+e^{(1+x)^{\frac{1}{x}}})]}{x}$
	\begin{solution}
		分子为两式相减,将1写成$\ln \ln e^e$,便可使用拉格朗日中值定理。\\
		得到:$1-\ln[\ln(x+e^{(1+x)^{\frac{1}{x}}})]$\\
		$=\ln \ln e^e-\ln[\ln(x+e^{(1+x)^{\frac{1}{x}}})]$\\
		$=\frac{1}{\xi\ln \xi}[e^e-(x+e^{(1+x)^{\frac{1}{x}}})](\xi \to e^e)$\\
		原式$=-\frac{1}{e^{e+1}}\lim_{x \to 0}(1+\frac{e^{(1+x)^{\frac{1}{x}}}-e^e}{x})$\\
		再次使用中值定理,可得$\lim_{x \to 0}\frac{e^{(1+x)^{\frac{1}{x}}}-e^e}{x}=\lim_{x \to 0}\frac{e^e((1+x)^\frac{1}{x})-e}{x}$,\\
		$(1+x)^{\frac{1}{x}}=e-\frac{ex}{2}+o(x)$,代入得:\\
		$\lim_{x \to 0}\frac{e^e((1+x)^\frac{1}{x})-e}{x}=-\frac{e^{e+!}}{2}$.\\
		故原式$=\frac{1}{2}-\frac{1}{e^{e+1}}$.
	\end{solution}
\end{example}

\begin{example}
	$\lim_{x \to 0}\frac{\tan \tan x-\sin \sin x}{\tan x-\sin x}$
	\begin{solution}
		此题分子不能直接使用中值定理,因此凑出一项。

		原式$=\lim_{x \to 0}\frac{\tan \tan x-\tan \sin x+\tan \sin x-\sin \sin x}{\frac{x^3}{2}}$\\
		$=\lim_{x \to 0}\frac{\tan \tan x-\tan \sin x}{\frac{x^3}{2}}+\lim_{x \to 0}\frac{\tan \sin x-\sin \sin x}{\frac{x^3}{2}}$\\
		$=\lim_{x \to 0}\frac{(sec^2\xi)^{'}(\tan x-\sin x)}{\frac{x^3}{2}}(\xi \to 0)+\lim_{x \to 0}\frac{\frac{1}{2}\sin^3x}{\frac{x^3}{2}}$\\
		$=1+1=2$.
	\end{solution}
\end{example}

关于该题的一个拓展:$\lim_{x \to 0}\frac{\overbrace{\tan \tan \cdots \tan x}^{n\text{个}\tan}-\overbrace{\sin \sin \cdots \sin x}^{n\text{个}\sin }}{\tan x-\sin x}$,这是一道套娃题,可以运用前面提到的差分法做,也可以运用数学归纳法得出套娃式子的泰勒展开,答案为n。

\section{常考题型}

\subsection{极限的存在}

\begin{example}
	设$a_n=\sqrt[n]{n}-\frac{(-1)^n}{n}(n=1,2,3,\cdots)$,则$a_n$(A)\xparen
	\begin{xchoices}[showanswer=true]
		\item* 有最大值,有最小值
		\item 有最大值,无最小值
		\item 无最大值,有最小值
		\item 无最大值,无最小值
	\end{xchoices}
	\vspace{0.3em}
	\begin{solution}
		易知$\lim_{x \to +\infty}a_n=1-0=1$,极限存在,有数列极限的定义,我们可以得到存在一个$N$,使得对任意$\epsilon>0$,当$n>N$时,$1-\epsilon<a_n<1+\epsilon$。

		注意N为一个很大的确切的数,$a_1=2,a_2=\sqrt{2}-\frac{1}{2}\approx 0.914,\cdots $,前N项一定有最大值和最小值,且最大值一定大于等于2,最小值一定小于等于0.914.\\而N之后的项,总是接近无限接近于1,由此我们可以得到此数列既有最大值也有最小值。
	\end{solution}
\end{example}

\begin{example}
	(1)已知$\lim_{x \to 0}[a\arctan\frac{1}{x}+(1+|x|)^{\frac{1}{x}}]$存在,求$a$的值.

	(2)$\lim_{x \to 0}(\frac{2+e^{\frac{1}{x}}}{1+e^{\frac{4}{x}}}+\frac{\sin x}{|x|})$.
\end{example}
\begin{solution}
	(1)$\lim_{x \to 0^-}[a\arctan\frac{1}{x}+(1+|x|)^{\frac{1}{x}}]=\lim_{x \to 0^-}[a\arctan\frac{1}{x}+((1-x)^{\frac{1}{-x}})^{-1}]=-\frac{\pi}{2}a+e^{-1}$,\\
	$\lim_{x \to 0^+}[a\arctan\frac{1}{x}+(1+|x|)^{\frac{1}{x}}]=\lim_{x \to 0^+}[a\arctan\frac{1}{x}+(1+x)^{\frac{1}{x}}]=\frac{\pi}{2}a+e$,\\
	左右极限相等,得$a=\frac{1-e^2}{\pi e}$.

	(2)$\lim_{x \to 0^+}(\frac{2+e^{\frac{1}{x}}}{1+e^{\frac{4}{x}}}+\frac{\sin x}{|x|})=0+1=1$.\\
	$\lim_{x \to 0^-}(\frac{2+e^{\frac{1}{x}}}{1+e^{\frac{4}{x}}}+\frac{\sin x}{|x|})=2-1=1$.\\
	左右极限皆等于1,故原极限等于1.
\end{solution}
\begin{remark}
	极限存在,当$x \to 0$时,在式子中含有$\arctan\frac{1}{x}$,$|x|$,$e^{\frac{1}{x}}$等项时,需要注意左极限和右极限要相等,因为当从左侧和右侧趋向于0时,这些式子的值是不等的。
\end{remark}

\begin{example}
	设$x \geq 0$时,$f(x)$满足$f^{'}(x)=\frac{1}{x^2+f^{2}(x)}$,且$f(0)=1$,证明:$\lim_{x \to +\infty}f(x)$存在。
	\begin{solution}
		由$f^{'}(x)=\frac{1}{x^2+f^{2}(x)}$可知,$f^{'}(x)>0$,$f(x)$单增,$f(x)>f(0)=1$,\\
		对$f^{'}(x)$积分,$$f(x)=f(0)+\int_{0}^{x}\frac{1}{x^2+f^{2}(x)}dx$$
		$$\leq 1+\int_{0}^{x}\frac{1}{x^2+1}dx$$
		$$=1+\arctan x-\arctan 0 \leq 1+\frac{\pi}{2}$$
		所以$f(x)$有上界,综上可得$\lim_{x \to +\infty}f(x)$存在.
	\end{solution}
\end{example}

\subsection{已知极限求参数}

\begin{example}
	设$\lim_{x \to \infty}\frac{n^{99}}{n^k-(n-1)^k}$存在且不为$0$,求常数$k$。
	\begin{solution}
		倒代换,$t=\frac{1}{n}, t \to 0$,$\lim_{t \to \infty}\frac{n^{99}}{n^k-(n-1)^k}=\frac{t^{k-99}}{1-(1-t)^k}$,又$1-(1-t)^k\sim tk$,故$\lim_{t \to 0}\frac{t^{k-99}}{tk}$存在且不为0,所以$k-99=1$,$k=100$。
	\end{solution}
\end{example}

\begin{example}
	设$a \geq 5$且为常数,则$k$为何值时极限$I=\lim_{x \to +\infty}[(x^{a}+8x^4+2)^k-x]$存在,并求此极限值。
	\begin{solution}
		倒代换,$t=\frac{1}{x},t \to 0^+$,$I=\lim_{t \to 0^+}\frac{(1+8t^{a-4}+2t^a)^k-t^{ak-1}}{t^{ak}}$,\\
		由于分母趋于零,所以分子也应趋于零,可得$ak-1\to 0,k=\frac{1}{a}$,\\
		故$I=\frac{(1+8t^{a-4}+2t^a)^{\frac{1}{a}}-1}{t}$\\
		$=\frac{1}{a}(8t^{a-5}+2t^a-1)	$\\
		故$I=\begin{cases}
				\frac{8}{5} , & a=5 \\
				0           , & a>5 \\
			\end{cases}$.
	\end{solution}
\end{example}

\begin{example}
	(1)设$\lim_{x \to 0}\frac{(\cos x-b)\sin x}{e^x-a}=3$,求$a,b$的值.

	(2)设$\lim_{x \to 0}\frac{\ln(3x^2-2x+1)+ax-bx^2}{x^2}=4$,求$a,b$的值.
	\begin{solution}
		(1)分子$(\cos x-b)\sin x\to 0$,要想极限存在,则分母也趋于零,得:$a=1$,\\
		将$a=1$代入,$\lim_{x \to 0}(\cos x-b)=3,b=-2$.

		(2)$\lim_{x \to 0}\frac{\ln(3x^2-2x+1)+ax-bx^2}{x^2}$\\
		$=\lim_{x \to 0}\frac{(3x^2-2x)-\frac{1}{2}(3x^2-2x)^2+ax-bx^2}{x^2}$\\
		$=\lim_{x \to 0}\frac{(a-2)x+(1-b)x^2}{x^2}=4$\\
		故$a=2,b=-3$.
	\end{solution}
\end{example}

\subsection{已知一个极限求另一个极限}
关于此类问题,一般都是通过等价无穷小替换,或者是泰勒展开,具体操作看下面几个例题。

\begin{example}
	设$\lim_{x \to 0}f(x)$存在,且$\lim_{x \to 0}\frac{\sqrt{1+f(x)\sin x}-1}{e^{2x}-1}=3$,求$\lim_{x \to 0}f(x)$。
	\begin{solution}
		当$x\to 0$时,$e^{2x}-1\to 0$,只有当$\sqrt{1+f(x)\sin x}-1 \to 0$时,极限存在.\\
所以$f(x)\sin x=0$,则$\sqrt{1+f(x)\sin x}-1\sim \frac{1}{2}f(x)\sin x$,又$e^{2x}-1\sim 2x$,代入得:\\
$\lim_{x \to 0}\frac{\frac{1}{2}f(x)\sin x}{2x}=3	$.
故$\lim_{x \to 0}\frac{f(x)}{4}=3,\lim_{x \to 0}f(x)=12$.
	\end{solution}
\end{example}

\begin{example}
	已知$\lim_{x \to 0}\frac{\sin 6x+xf(x)}{x^3}=0$,求$\lim_{x \to 0}\frac{6+f(x)}{x^2}$。
	\begin{solution}
		\begin{align*}
			\text{原式}&=\lim_{x \to 0}\frac{6x+xf(x)}{x^3}\\
			&=\lim_{x \to 0}\frac{6x-\sin 6x+\sin 6x+xf(x)}{x^3}\\
			&=\lim_{x \to 0}\frac{6x-\sin 6x}{x^3}\\
			&=\lim_{x \to 0}\frac{\frac{1}{6}(6x)^3}{x^3} = 36
		\end{align*}
	\end{solution}
\end{example}

\begin{example}
	已知$\lim_{x \to 0}(1+x+\frac{f(x)}{x})^{\frac{1}{x}}=e^3$,求$\lim_{x \to 0}\frac{f(x)}{x^2}$。
	\begin{solution}
		\begin{align*}
			\lim_{x \to 0}(1+x+\frac{f(x)}{x})^{\frac{1}{x}} &= e^{\lim_{x \to 0}\frac{1}{x}\ln(1+x+\frac{f(x)}{x})}\\
			&=e^{\lim_{x \to 0}\frac{1}{x}(x+\frac{f(x)}{x})}\\
			&=\lim_{x \to 0}e^{1+\frac{f(x)}{x^2}}=e^3
		\end{align*}
		故$\lim_{x \to 0}\frac{f(x)}{x^2}=2$。
	\end{solution}
\end{example}

\begin{remark}
	运用等价无穷小时需要注意取值,此题中$\ln (1+x+\frac{f(x)}{x})\sim x+\frac{f(x)}{x}$,因为极限$\lim_{x \to 0}\frac{1}{x}\ln(1+x+\frac{f(x)}{x})$存在,分母趋于零,所以分子也趋于零,这样我们才能够使用等价无穷小。
\end{remark}

\subsection{含有抽象函数的极限}
此类问题,需要着重注意题干的条件,往往需要用到导数的定义,而不能盲目运用洛必达法则求导。

\begin{example}
	已知$f(0)=0,f^{'}(0)=2$,求$\lim_{x \to +\infty}[f(\frac{1}{n^2})-\frac{1}{n^2}+1]^{3n^2}$。
	\begin{solution}
		遇到指数,取对数,则$\lim_{x \to +\infty}[f(\frac{1}{n^2})-\frac{1}{n^2}+1]^{3n^2}=\lim_{n \to \infty}e^{3n^2[f(\frac{1}{n^2})-\frac{1}{n^2}+1]}$。

$3n^2[f(\frac{1}{n^2})-\frac{1}{n^2}+1]\sim3n^2f(\frac{1}{n^2})-3$.

由导数的定义,可知$n^2f(\frac{1}{n^2})=\lim_{n \to \infty}\frac{f(\frac{1}{n^2})-f(0)}{\frac{1}{n^2}-0}=f^{'}(0)$.

所以原式$=3f^{'}(0)-3=3$.
	\end{solution}
\end{example}

\begin{example}
	已知$f(x)>0,f^{'}(0)$存在,求$\lim_{x \to 0}\frac{f(x)^{f(x)}-f(0)^{f(0)}}{x}$。
	\begin{solution}
		记$G(x)=f(x)^{f(x)},$\\
则$\lim_{x \to 0}\frac{f(x)^{f(x)}-f(0)^{f(0)}}{x}=\lim_{x \to 0}\frac{G(x)-G(0)}{x-0}=G^{'}(x)|_{x=0}$\\
对$G(x)$求导($G^{'}(x)=(e^{f(x)\ln f(x)})^{'}$),\\
可得原式$=G^{'}(0)=f(0)^{f(0)}f^{'}(0)(\ln f(0)+1)$.
	\end{solution}
\end{example}

\begin{example}
	设$f(x)$二阶连续可导,$f(0)=0,f^{'}(0)=0,f^{''}(0)=6$,求$\lim_{x \to 0}\frac{f(\sin ^2 x)}{x^4}$。
\end{example}
\begin{solution}
	\begin{enumerate}
		\item 法1(泰勒展开)
		由题给条件,我们很容易想到使用泰勒公式在$x=0$处展开,
得$$f(x)=f(0)+f^{'}(0)x+\frac{f^{''}(0)}{2!}x^2+o(x^2),$$
可得$f(x)\sim 3x^2$,代入可得$\lim_{x \to 0}\frac{f(\sin ^2 x)}{x^4}=\lim_{x \to 0}\frac{3\sin^4x}{x^4}=3$.
		\item 法2(洛必达法则)
		当然本题也可以利用洛必达法则,\\
$\lim_{x \to 0}\frac{f(\sin ^2 x)}{x^4}$\\
$=\lim_{x \to 0}\frac{2\sin x\cos xf^{''}(\sin^2x)}{4x^3}=\lim_{x\to 0}\frac{f^{'}(\sin^2x)}{2x^2}$\\
$=\lim_{x \to 0}\frac{2\sin x\cos xf^{''}(\sin^2x)}{4x}=\lim_{x \to 0}\frac{f^{''}(\sin^2x)}{2x}=3$.
	\end{enumerate}
\end{solution}

\begin{example}
	设$f(x)$在$x=0$的邻域内二阶可导,且$f^{'0}=0$,试计算$\lim_{x \to 0}\frac{f(x)-f(\ln(1+x))}{x^3}$。
	\begin{solution}
		分子为两式相减,考虑到运用拉格朗日中值定理,$f(x)-f(\ln(1+x))=f^{'}(\xi)(x-\ln (1+x))\sim f^{'}(\xi)\cdot \frac{1}{2}x^2(\xi\text{介于}x\text{与}\ln (1+x)\text{之间})$.\\
$\lim_{x \to 0}\frac{f(x)-f(\ln(1+x))}{x^3}=\lim_{x \to 0}\frac{f^{'}(\xi)}{2x}$\\
$=\lim_{x \to 0}\frac{1}{2}\frac{f^{'}(x)-f^{'}(0)}{x-0}$\\
由导数的定义,得原式$=\frac{f^{''}(0)}{2}$。
	\end{solution}
\end{example}

\subsection{递推数列的极限}

递推数列的极限在我校往年期中考试中多有考查,当然其难度都不大,但不排除会考查难度较大的题目,因此需多加注意.

该类题型可以大致分为两种,一是可以直接求出通项公式,二是不可求出通项公式或者很难求通项公式的类型.第一类问题在高中我们就能够轻松解决,这里主要讨论第二种类型,对于此类型的题目,我们往往通过单调有界准则,压缩映像原理来解决.

\subsubsection{运用单调有界准则}

单增有上界的数列收敛,单减有下界的数列收敛.

\begin{example}
	设$x_1>0,x_{n+1}=\frac{1}{2}(x_n+\frac{1}{x_n}),n=1,2,\cdots $.证明$\lim_{n \to \infty}x_n$存在,并求此极限。
	\begin{proof}
		由题给条件可知$a_n>0$,基本不等式放缩可得:$x_{n+1}\geq 1$,\\
又$\frac{x_{n+1}}{x_n}=\frac{1}{2}(1+\frac{1}{x_{n}^{2}})\geq\frac{1}{2}(1+\frac{1}{1})=1$,故数列单减有下界,所以数列收敛,极限存在.对等式$x_{n+1}=\frac{1}{2}(x_n+\frac{1}{x_n})$两端取极限,令$\lim_{n \to \infty}x_n=a$,有$a=\frac{1}{2}(a+\frac{1}{a})$,解得$a=1$.
	\end{proof}
\end{example}

\begin{example}
	设数列${x_n}$满足$\ln x_n+\frac{1}{x_{n+1}}<1$,证明$\lim_{n \to \infty}x_n$存在,并求此极限。
	\begin{proof}
		由高中知识我们不难得到:$\ln x+\frac{1}{x}\geq1$,结合所给条件可以得到:$\ln x_n+\frac{1}{x_{n+1}}<1\geq\ln x_n+\frac{1}{x_n}$,于是可以得到:$x_n<x_{n+1}$.\\
由$\ln x_n+\frac{1}{x_{n+1}}<1$可以得到$\ln x_n<1$,即$x_n<e$,\\
因此数列单增有上界,故极限存在.\\
由$\ln a+\frac{1}{a}=1$,$a=1$,则$\lim_{n \to \infty}x_n=1$。
	\end{proof}
\end{example}

\begin{example}
	设$x_1>0,x_{n+1}=\ln(1+x_n),n=1,2,\cdots$,证明:$\lim_{n \to \infty}x_n=0$。
	\begin{proof}
		易知$x_n>0$,当然也可以用数学归纳法严谨简洁地证明,证明过程如下:

$n=1$时,$x_1>0$成立,假设$n=k$时成立,有$x_k>0$,则$n=k+1$时,$x_{k+1}=\ln(x_k+1)>0$,即证。

不难证明$x_{n+1}=\ln(1+x_n)<x_n$,则数列单减有下界。从而可以得到极限存在,并求得极限$\lim_{n \to \infty}x_n=0$。
	\end{proof}
\end{example}
\textbf{例3:}

\textit{解:}



\subsubsection{运用压缩映像原理}

(1)对任一数列${x_n},$若存在$0\leq k<1,$使得$|x_{n+1}-x_n|\leq k|x_n-x_{n-1}|$成立,则${x_n}$一定收敛,

\begin{proof}
	$|x_{n+1}-x_n|\leq k|x_n-x_{n-1}|\leq \cdots \leq k^{n-1}|x_2-x_1|$,由此可得$\lim_{n \to +\infty}|x_{n+1}-x_n|=0$,那么$\lim_{n \to +\infty}x_{n+1}=\lim_{n \to +\infty}x_n$,所以${x_n}$极限存在\mn{或由$x_n=x_1+\sum_{k=1}^{n-1}(x_{k+1}-x_k)$说明极限存在}。
\end{proof}

(2)设$x_{n+1}=f(x_n)$,若存在$0\leq k<1$,使得$|f^{'}(x)|\leq k<1$成立,则${x_n}$一定收敛。

解题套路:

方法1:直接作差,$|x_{n+1}-x_n|=|f(x_n)-f(x_{n-1})|=|f^{'}(\xi _n)||x_n-x_{n-1}|\leq k|x_n-x_{n-1}|$。

方法2:先求出$x=f(x)$的根,若极限存在,则该根就为数列极限,设根为$r$,则$|x_{n+1}-r|=|f(x_n)-f(r)| =|f^{'}(\xi _n)||x_n-r|\leq k|x_n-x_{n-1}|\leq k^2|x_{n-1}-x_{n-2}|\leq \cdots \leq k^{n-1}|x_2-x_1|$.因为$0<k<1$,所以$\lim_{n \to \infty}k^{n-1}|x_2-x_1|=0,$可得$\lim_{n \to \infty}x_{n+1}=r$。

\begin{example}
	设$x_1<-1,x_{n+1}+\sqrt{1-x_n}=0,n=1,2,\cdots.$证明$\lim_{n \to \infty}x_n$存在,并求此极限。
	\begin{solution}
		由题给条件$x_{n+1}+\sqrt{1-x_n}=0$不难得到$x_n<0$,$x_{n+1}-x_n=\sqrt{1-x_{n-1}}-\sqrt{1-x_n}=\frac{x_n-x_{n-1}}{\sqrt{1-x_{n-1}}+\sqrt{1-x_n}}$,分母恒大于2,放缩后根据压缩映像原理可得到数列极限存在,并可求出极限为$-\frac{\sqrt{5}+1}{2}$。
	\end{solution}
\end{example}

\begin{example}
	设$x_1=2,x_{n+1}=2+\frac{1}{x_n},n=1,2,\cdots$,求极限$\lim_{n \to \infty}x_n$。
	\begin{solution}
		由题给条件可知$x_n\geq 2$,根据解题套路中的方法1和方法2都可以求解,答案为$\sqrt{2}+1$。
	\end{solution}
\end{example}
