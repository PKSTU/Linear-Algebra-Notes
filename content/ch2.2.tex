\chapter{逆矩阵}\label{ch:2}

在上一节课中,我们已经学习了矩阵的加法,减法,乘法.根据经验,我们想矩阵是否存在类似的"除法".如果矩阵存在"除法",那么对于一个线性方程组$\mathbf{A}\mathbf{x}=\mathbf{b}$,我们是否可以把$\mathbf{A}$放到等式右端,解出线性方程组的解.本节课我们所研究的逆矩阵就可以帮助我们解决此类问题.

本节课的主要内容包括:
\begin{enumerate}
	\item 逆矩阵与伴随矩阵的概念;
	\item 矩阵可逆的条件;
	\item 逆矩阵的性质
\end{enumerate}

首先,我们给出逆矩阵的定义.
\begin{definition}[\textbf{逆矩阵}]
	设$\mathbf{A}$为$n$阶方阵,如果存在$n$阶方阵$\mathbf{B}$,使得
	\[
		\mathbf{AB}=\mathbf{BA}=\mathbf{I},
	\]
	则称方阵$\mathbf{A}$是\textbf{可逆的},或者称为\textbf{非奇异的},并称方阵$\mathbf{B}$为方阵$\mathbf{A}$的\textbf{逆矩阵},记为$\mathbf{A}^{-1}$,即$\mathbf{A}^{-1}=\mathbf{B}$.

	不存在逆矩阵的方阵称为\textbf{奇异矩阵}
\end{definition}

事实上,在这节课的概述部分,我们假设矩阵的"除法"是不正确的,矩阵不存在除法运算,我们需要通过逆矩阵的方式来回避这种运算.例如,我们不能把$\mathbf{A}^{-1}$写成$\frac{1}{\mathbf{A}}$.

我们再引入伴随矩阵的定义.
\begin{definition}[\textbf{伴随矩阵}]
	设$\mathbf{A}=(a_{ij})_{n\times n}$为$n(n\ge2)$阶方阵,$\det(\mathbf{A})$的元素$a_{ij}$的代数余子式为$\mathbf{A}_{ij}(i,j=1,2,\cdots,n)$,则称以$\mathbf{A}_{ji}$为$\left(i,j\right)$元素的$n$阶方阵为$\mathbf{A}$的伴随矩阵,记为$\mathbf{A}^{\ast}$,即
	\[
		\mathbf{A}^{\ast}=
		\begin{bmatrix}
			\mathbf{A}_{11} & \mathbf{A}_{21} & \cdots & \mathbf{A}_{n1} \\
			\mathbf{A}_{12} & \mathbf{A}_{22} & \cdots & \mathbf{A}_{n2} \\
			\vdots          & \vdots          & \ddots & \vdots          \\
			\mathbf{A}_{1n} & \mathbf{A}_{2n} & \cdots & \mathbf{A}_{nn}
		\end{bmatrix}  .
	\]
\end{definition}
\begin{example}
	\[
		\mathbf{A} = \begin{bmatrix}
			a_{11} & a_{12} & a_{13} \\
			a_{21} & a_{22} & a_{23} \\
			a_{31} & a_{32} & a_{33}
		\end{bmatrix}
	\]的伴随矩阵是
	\[
		\mathbf{A}^{\ast} = \begin{bmatrix}
			+\begin{vmatrix} a_{22} & a_{23} \\ a_{32} & a_{33} \end{vmatrix} &
			-\begin{vmatrix} a_{12} & a_{13} \\ a_{32} & a_{33} \end{vmatrix} &
			+\begin{vmatrix} a_{12} & a_{13} \\ a_{22} & a_{23} \end{vmatrix}       \\
			                                                                  &   & \\
			-\begin{vmatrix} a_{21} & a_{23} \\ a_{31} & a_{33} \end{vmatrix} &
			+\begin{vmatrix} a_{11} & a_{13} \\ a_{31} & a_{33} \end{vmatrix} &
			-\begin{vmatrix} a_{11} & a_{13} \\ a_{21} & a_{23} \end{vmatrix}       \\
			                                                                  &   & \\
			+\begin{vmatrix} a_{21} & a_{22} \\ a_{31} & a_{32} \end{vmatrix} &
			-\begin{vmatrix} a_{11} & a_{12} \\ a_{31} & a_{32} \end{vmatrix} &
			+\begin{vmatrix} a_{11} & a_{12} \\ a_{21} & a_{22} \end{vmatrix}
		\end{bmatrix}.
	\]
\end{example}

关于伴随矩阵,有以下结论:
\begin{theorem}
	设$\mathbf{A}$为$n(n\ge2)$阶 方阵,则成立
	\[
		\mathbf{AA^{\ast}}=\mathbf{A^{\ast}A}=\det (\mathbf{A})\mathbf{I}.
	\]
\end{theorem}
由此可以得到以下结论:
\begin{theorem}
	设$\mathbf{A}$为$n(n\ge2)$阶方阵,$\det (\mathbf{A})\ne 0$,则
	\[
		\det(\mathbf{A}^{\ast})=(\det(\mathbf{A}))^{n-1}.
	\]
\end{theorem}
至此,我们已经可以求可逆矩阵的逆矩阵.下面,我们将严格给出矩阵可逆的充分必要条件.
\begin{theorem}[\textbf{方阵可逆的充分必要条件}]
	$n(n\ge2)$阶矩阵$\mathbf{A}$可逆的充分必要条件是$\det(\mathbf{A})\ne0$.且当$\mathbf{A}$可逆时,有
	\[
		\mathbf{A}^{-1}=\frac{1}{\det(\mathbf{A})}\mathbf{A}^{\ast}.
	\]
\end{theorem}

\begin{example}
	判断矩阵$\mathbf{A} = \begin{bmatrix} {{\cos\theta}} & {{-\sin\theta}}\\ {{\sin\theta}}  & {{\cos\theta}} \end{bmatrix}$是否可逆,若可逆,求其逆矩阵.
\end{example}
\begin{solution}
	由于$\det(\mathbf{A})=1\ne0,$故$\mathbf{A}$可逆,

	$
		\mathbf{A}^{-1}=\frac{1}{\det(\mathbf{A})}\mathbf{A}^{\ast}=\begin{bmatrix} {{\cos\theta}} & {{\sin\theta}}\\ {{-\sin\theta}}  & {{\cos\theta}} \end{bmatrix}.
	$
\end{solution}

\begin{example}
	对于对角阵$\mathbf{D}=diag(d_1,d_2,\cdots,d_n)$,证明:$\mathbf{D}$可逆当且仅当$\mathbf{D}$的主对角线元素$d_1,d_2,\cdots,d_n$均不为零.且当$\mathbf{D}$可逆时,有$\mathbf{D}^{-1}=diag(d_{1}^{-1},d_{2}^{-1},\cdots,d_{n}^{-1}).$
\end{example}
\begin{proof}
	由于$\det(\mathbf{D})=d_1d_2\cdots d_n$,

	所以$\mathbf{D}$可逆

	当且仅当$\det(\mathbf{D})\ne0$

	当且仅当$d_1,d_2,\cdots,d_n$均不为零.

	此时,$\mathbf{D}^{-1}=\frac{1}{\det(\mathbf{D})}\mathbf{D}^{\ast}=\frac{1}{d_1d_2\cdots d_n}\cdot diag(\frac{\det(\mathbf{D})}{d_1},\frac{\det(\mathbf{D})}{d_2},\cdots,\frac{\det(\mathbf{D})}{d_n})=diag(d_{1}^{-1},d_{2}^{-1},\cdots,d_{n}^{-1}).$
\end{proof}

逆矩阵可以用来证明Cramer法则,求解线性方程组.


\begin{example}
	利用逆矩阵的方法求解线性方程组:
	\[
		\left\{\begin{matrix}
			x_1+2x_2+x_3=1,   \\
			2x_1+5x_2+4x_3=0, \\
			x_1+x_2=1.
		\end{matrix}\right.
	\]
\end{example}
\begin{solution}
	\textbf{第一步}:把方程组写成矩阵形式:
	\[
		\mathbf{A}\mathbf{x}=\mathbf{b},
	\]

	其中
	\[
		\mathbf{A} = \begin{bmatrix}
			1 & 2 & 1 \\
			2 & 5 & 4 \\
			1 & 1 & 0
		\end{bmatrix},
		\mathbf{x}=\begin{bmatrix}
			x_{1} \\
			x_{2} \\
			x_{3}
		\end{bmatrix},
		\mathbf{b}=\begin{bmatrix}
			1 \\
			0 \\
			1
		\end{bmatrix}.
	\]

	\textbf{第二步}:计算$\det(\mathbf{A})\ne0$,故$\mathbf{A}$可逆,线性方程组有唯一解
	\[
		\mathbf{x}=\begin{bmatrix}
			x_{1} \\
			x_{2} \\
			x_{3}
		\end{bmatrix}=\mathbf{A}^{-1}\mathbf{b}=\begin{bmatrix}
			1 & 2 & 1 \\
			2 & 5 & 4 \\
			1 & 1 & 0
		\end{bmatrix}^{-1}\begin{bmatrix}
			1 \\
			0 \\
			1
		\end{bmatrix}=\begin{bmatrix}
			-1 \\
			2  \\
			-2
		\end{bmatrix}.
	\]
\end{solution}

利用逆矩阵也可以来求解矩阵方程.
\begin{example}
	设$\mathbf{A},\mathbf{B}$是可逆矩阵,则矩阵方程$\mathbf{AX=C,XB=D,AXB=F}$的解分别为
	\[
		\mathbf{X=A^{-1}C,X=DB^{-1},X=A^{-1}FB^{-1}}.
	\]
\end{example}

从以上这些例子我们可以看出:如果一个矩阵可逆,那么它就具有很好的性质,可以按照代数运算的方法处理问题.

下面介绍一些可逆矩阵的性质.

设$\mathbf{A},\mathbf{B}$为同阶可逆方阵,常数$k\ne0$,则有:
\begin{enumerate}
	\item $\mathbf{A}^{-1}$可逆,且$(\mathbf{A}^{-1})^{-1}=\mathbf{A}$;
	\item $\mathbf{A}^T$可逆,且$(\mathbf{A}^T)^{-1}=(\mathbf{A}^{-1})^{T};$
	\item $k\mathbf{A}$可逆,且$(k\mathbf{A})^{-1}=\frac{1}{k}\mathbf{A}^{-1}$;
	\item $\mathbf{AB}$可逆,且$(\mathbf{AB})^{-1}=\mathbf{B}^{-1}\mathbf{A}^{-1}$;
	\item $\det(\mathbf{A}^{-1})=\frac{1}{\det(\mathbf{A})}.$
\end{enumerate}

\begin{example}
	设$\mathbf{A}$为$3$阶矩阵,$\det(\mathbf{A})=\frac{1}{2},\mathbf{A}^{\ast}$为$\mathbf{A}$的伴随矩阵.求行列式$D=\det((3\mathbf{A})^{-1}-2\mathbf{A}^{\ast})$的值.
\end{example}
\begin{solution}
	\[
		D=\det((3\mathbf{A})^{-1}-2\mathbf{A}^{\ast})=\det(\frac{1}{3}\mathbf{A}^{-1}-2\det(\mathbf{A})\mathbf{A}^{-1})=\det(-\frac{2}{3}\mathbf{A}^{-1})=(-\frac{2}{3})^3\frac{1}{\det(\mathbf{A})}=-\frac{16}{27}.
	\]
\end{solution}