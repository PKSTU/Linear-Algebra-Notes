\RequirePackage{fix-cm}
\documentclass[10pt]{article}
\usepackage{tikz}
\usetikzlibrary{arrows.meta}
\usetikzlibrary{arrows,shapes.arrows}

\usepackage[scale=1,,width=8in,height=10in]{geometry}
\usepackage[T1]{fontenc}
\usepackage{inputenc}
\usepackage[no-math]{fontspec}
\usepackage{indentfirst}
\setlength\parindent{2em}
\usepackage{amsmath}
\usepackage{amssymb}
\usepackage{amsfonts}
\usepackage[math-style=TeX,
            bold-style=ISO]{unicode-math}
\setmathfont{XITS Math}
\setmathfont{XITS Math Bold}[version=bold]
\usepackage[bb=ams,
            scr=rsfs]{mathalpha}
\everymath{\displaystyle}
\allowdisplaybreaks[4]
\DeclareSymbolFont{largesymbols}{OMX}{cmex}{m}{n}
\DeclareMathSymbol{\msum}{\mathop}{largesymbols}{"50}
\DeclareSymbolFont{letters}{OML}{cmm}{m}{it}
\DeclareMathSymbol{\mpi}{\mathord}{letters}{"19}
\renewcommand{\boldsymbol}{\symbf}
\renewcommand{\mathscr}{\symscr}
\usepackage{ctex}
\xeCJKsetup{PlainEquation=true}
\xeCJKsetup{CJKmath=true}
\xeCJKsetup{EmboldenFactor=2.5}
\newcommand{\lbtitlefont}{\CJKfontspec[AutoFakeBold=true]{方正清刻本悦宋简体}}
\setmainfont{XITS}
\usepackage{fancyhdr}
\usepackage{xcolor}
\definecolor{coverbgcolor}{RGB}{138,1,1}
\pagecolor{coverbgcolor}

\begin{document}
\let\pi\mpi
\let\sum\msum

\pagestyle{empty}
\fancyhf{}
%% 封面背景数学公式模糊
%  本质上是做一个阴影投影
\pgfmathsetmacro{\shadowangle}{180}        % 阴影角度
\newlength{\shadowdistance}                
\pgfmathsetlength{\shadowdistance}{0pt}    % 距离
\pgfmathsetmacro{\shadowopacity}{0.7}      % 透明度
\pgfmathtruncatemacro{\totshadow}{100}     % 透明度总数
\pgfmathsetmacro{\shadowspread}{0.04}      % 扩散程度
\pgfmathsetmacro{\shadowsize}{0.1}         % 阴影大小
\newcommand{\blurnode}[2]{                 % \blurnode{字号}{内容}
    \path[opacity={\shadowopacity/\totshadow},shift={({\shadowangle-180}:\shadowdistance)},scale={1+\shadowsize}]
    foreach \nshadow [evaluate=\nshadow as \angshadow using \nshadow/\totshadow*360] in {1,...,\totshadow}{
    node[above right,inner sep=0pt,outer sep=0pt] at (\angshadow:\shadowspread) {
        \color{white}\fontsize{#1}{#1}\selectfont #2}};
}
%%
\begin{tikzpicture}[overlay,remember picture]
    \draw[coverbgcolor!50,opacity=0.7,step=5mm] (current page.north west) grid (\paperwidth,-\paperheight);
    \draw[coverbgcolor] (0.45\paperwidth,-0.5\paperheight) node {\fontsize{650}{25}\selectfont $\symscr{A}$};
    \draw[white] (0.78\paperwidth,-0.4\paperheight) node[align=center] (M) {
        \fontsize{20}{25}\selectfont{\lbtitlefont 可视化方法}\\
        \fontsize{10}{25}\selectfont{Visual Matrix\, {$\symscr{A}$}nalysis}
        % \hspace{4.5em}\zihao{3} J. Wang
        };
    \draw[white] (M.west) node[left] {\fontsize{40}{40}\selectfont{\lbtitlefont 矩阵分析}};
    \begin{scope}[xshift=0.5\paperwidth,yshift=-0.7\paperheight, ultra thick, opacity=0.5]
        \draw[white,-Stealth] (0,-2)--(0,5);
        \draw[white,-Stealth] (-2,0)--(8,0);
        \draw[white] (0,2.5) circle (0.8);
        \draw[white] (3,0) circle (1.2);
        \draw[white] (5,0) circle (1.2);
        \fill[white] (0,2.5) circle (3pt);
        \fill[white] (3,0) circle (3pt);
        \fill[white] (5,0) circle (3pt);
    \end{scope}
    \begin{scope}[xshift=0.045\paperwidth,yshift=-0.6\paperheight]
        \blurnode{20}{
            $
            \boldsymbol{J}=\left[\begin{array}{llll}
            \boldsymbol{J}_{1} & & & \\
            & \boldsymbol{J}_{2} & & \\
            & & \ddots & \\
            & & & \boldsymbol{J}_{n}
            \end{array}\right]
            $
        }
    \end{scope}
    \begin{scope}[xshift=0.045\paperwidth,yshift=-0.3\paperheight]
        \blurnode{20}{
            $
            \mathrm{e}^{\symbf{A}}=\boldsymbol{I}+\boldsymbol{A}+\frac{\boldsymbol{A}^{2}}{2 !}+\cdots+\frac{\boldsymbol{A}^{k}}{k !}+\cdots
            $
        }
    \end{scope}
    \begin{scope}[xshift=0.5\paperwidth,yshift=-0.2\paperheight]
        \blurnode{20}{
            $
            \begin{aligned}
                \sin \boldsymbol{A}    & =\msum_{k=0}^{+\infty}(-1)^{k} \frac{\boldsymbol{A}^{2 k+1}}{(2 k+1) !} \\
                \cos \boldsymbol{A}    & =\msum_{k=0}^{+\infty}(-1)^{k} \frac{\boldsymbol{A}^{2 k}}{(2 k) !}
            \end{aligned}
            $
        }
    \end{scope}
    \begin{scope}[xshift=0.005\paperwidth,yshift=-0.9\paperheight]
        \blurnode{20}{
            $
            \begin{aligned}
                \varphi_{k}(x)= & \frac{\varphi_{\symbf{A}}(x)}{\left(x-\lambda_{k}\right)^{d_{k}}}\\
                              = & \left(x-\lambda_{1}\right)^{d_{1}} \cdots\left(x-\lambda_{k-1}\right)^{d_{k-1}} \cdot\left(x-\lambda_{k+1}\right)^{d_{k+1}} \ldots\left(x-\lambda_{s}\right)^{d_{s}} 
            \end{aligned}
            $
        }
    \end{scope}
\end{tikzpicture}
\end{document}

