\chapter*{前言}
\addcontentsline{toc}{chapter}{前言}

\plainsection*{献给读者}

西安交通大学的学弟学妹们: 你们好!

\subsubsection*{你为何收到这份讲义}
秋风生渭水,落叶满长安,欢迎你们来到古都西安。在飒爽的季节里见到你们,我们的记忆里常常涌起初到交大时的激动与迷茫。各个社团与活动与学业压力一同到来,每个人都在尝试着平衡学习与生活——这是大学生活天平的两端。交大是古城西安中文化底蕴最深厚,最务实的学校之一,而作为交大学子的你们在大一面对的最重要的科目之一就是线性代数,在未来的许多专业课程中,线性代数和高等数学的基础知识都会发挥非常重要的作用。\textbf{彭康学业辅导与发展中心(彭康学导团)}特定准备了一份高等数学引导讲义,带你理解高数学习的要领。另外,欢迎没有加入QQ群\textbf{彭小帮2.0(397499749)}的同学前来交流学习,这是交大进行学习交流的基地之一,有众多同辈同学与前辈的学长学姐与你共同学习。

\subsubsection*{你应该如何理解这份讲义}
这份讲义并不是严格标准的教学用书,只是一份引导性质的讲义。这份讲义\textbf{不可以替代上课听讲},但是可以帮助你课前预习与课后巩固。这份讲义是学长学姐的感悟与体验,但是你需要整理属于你自己的学习心得。\textbf{这份讲义的版权属于西安交通大学彭康学业辅导与发展中心,不可售卖,不可复印出售。}

\subsubsection*{当你学数学的时候,你在学习什么}
学习数学的时候,我们通常有两套语言在彼此交互:\textbf{纸面上的数学语言与脑海中的自然语言}。而许多同学没有学习好数学的根本原因,是困扰于数学语言而不知道某知识点的自然语言如何表述,这就导致在学习的过程中出现了困顿与阻滞的状态。因此,这份讲义\textbf{花开两朵,各表一枝},既注重数学知识与公式的引导,也强调使用自然语言说明清楚某个知识点的内涵与应用。

\subsubsection{本讲义的编写与阅读原则}

本讲义的编写按照《线性代数与解析几何》课本的顺序,上下梳理三部分知识点,注重知识点的连续与衍生应用。\textbf{轻推导,重实践},秉持“让交大每位学生顺利走入线代殿堂”的原则,左右横揽重点知识点与相关衍生习题。对于每一节的内容,\textbf{第一部分是知识点概览},会提纲挈领的告诉你学习这一节你需要掌握的内容与重点。\textbf{第二部分是知识点的背景叙述},帮助你复习预备知识,使各知识点彼此贯通。\textbf{第三部分会讲述具体知识点},会按照自然语言引导+数学语言结论的方式讲述知识点,是每一节的重中之重。\textbf{第四部分放在最后,是相关结论、二级思维、套路与习题结构的整合之处}。

\subsubsection{结束语}
怕什么真理无穷,进一步有进一步的欢喜。欢迎各位同学加入彭康学业辅导与发展中心,彭康书院的东19-114室是我们的线下大本营,里面有专门答疑的志愿者为你们答疑。此外,线上学习研究中心的基地是QQ群:彭小帮2.0(397499749),欢迎各位同学加入,我们永远欣喜于你们的到来。

\plainsection*{笔误及疏漏}

由于水平有限,编写组成员对本书中可能出现的错误深表歉意,并希望读者予以批评指正。

\vspace{3em}
\hfill\begin{minipage}{7cm}
	\begin{flushright}\kaiti
		彭康书院学业辅导与发展中心·志愿者部\\
		\zhtoday
	\end{flushright}
\end{minipage}